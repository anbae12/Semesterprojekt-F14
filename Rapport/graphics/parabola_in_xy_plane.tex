\draw[->] (-22,0)--(32,0) node[right]{Y [m]};
\draw[<-] (0,-22)--(0,12)node[right]{X [m]};
%%%%%%%%%%%%%%%%%%%%%%%%%%
\draw(-0.2,5)node[left]{-5}--(0.2,5);
\draw(-0.2,10)node[left]{-10}--(0.2,10);
\draw(-0.2,-5)node[left]{5}--(0.2,-5);
\draw(-0.2,-10)node[left]{10}--(0.2,-10);
\draw(-0.2,-15)node[left]{15}--(0.2,-15);
\draw(-0.2,-20)node[left]{20}--(0.2,-20);
\draw(5,-0.2)node[below]{5}--(5,0.2);
\draw(10,-0.2)node[below]{10}--(10,0.2);
\draw(15,-0.2)node[below]{15}--(15,0.2);
\draw(20,-0.2)node[below]{20}--(20,0.2);
\draw(25,-0.2)node[below]{25}--(25,0.2);
\draw(30,-0.2)node[below]{30}--(30,0.2);
\draw(-5,-0.2)node[below]{-5}--(-5,0.2);
\draw(-10,-0.2)node[below]{-10}--(-10,0.2);
\draw(-15,-0.2)node[below]{-15}--(-15,0.2);
\draw(-20,-0.2)node[below]{-20}--(-20,0.2);
%%%%%%%%%%%%%%%%%%%%%%%%%%%
%%% If we want to change it back..
% \newcommand{\arcThroughThreePoints}[4][]{
% \coordinate (middle1) at ($(#2)!.5!(#3)$);
% \coordinate (middle2) at ($(#3)!.5!(#4)$);
% \coordinate (aux1) at ($(middle1)!1!90:(#3)$);
% \coordinate (aux2) at ($(middle2)!1!90:(#4)$);
% \coordinate (center) at ($(intersection of middle1--aux1 and middle2--aux2)$);
% \draw[#1] 
%  let \p1=($(#2)-(center)$),
%       \p2=($(#4)-(center)$),
%       \n0={veclen(\p1)},       % Radius
%       \n1={atan2(\x1,\y1)}, % angles 
%       \n2={atan2(\x2,\y2)}, 
%       \n3={\n2>\n1?\n2:\n2+360}       
%     in (#2) arc(\n1:\n3:\n0);
% }
% \arcThroughThreePoints{-18.29,-5.5}{0,-19.1}{18.29,-5.5}

\draw (18.29,-5.5) arc(163:17:-19.1); %this is simpler and works on all machines...

%%%%%%%%%%%%%%%%%%%%%%%%%%%%%
\fill (0,0) circle[radius=9pt] node[above]{TCS};
\fill (0,-19.1) circle[ radius=9pt] node[below, right]{B};
\fill (-18.29,-5.5) circle[ radius=9pt];
\draw[red] (-18.29,-5.5) node[above,black]{D}--(30.54,9.19) node[above,black]{G};
\draw(-20,-5.5)--(20,-5.5);
\draw(18.29,-5.5)--(0,0);
\fill (19.4793,5.9) circle[radius=9pt] node[above]{SB};
\fill (30.54,9.19) circle[radius=9pt];
\coordinate (PP2) at (20,3.5);
\coordinate (PP1) at (20,0);
\draw (PP1) arc (0:16.9:20);
\draw (PP2)--(PP2) node[right]{ \(\alpha =15,872^{\circ}\) };