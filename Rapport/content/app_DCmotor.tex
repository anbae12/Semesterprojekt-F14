\section{DC motoren}
Dette appendix beskæftiger sig med bestemmelsen af motorparametrene.
\subsection{DC Motor karakteristik}
DC Motoren kan modelleres efter diagrammet på fig. \todo[inline]{kredsløbsdiagram skal indsættes}\todo[inline]{Kildehenvisning til DC motor model}.
Ligningen for $V_m$ er givet ved ligning \ref{eq:Vm_transient0}.
\begin{equation}
	\label{eq:Vm_transient0} 
	V_m(t)=L_m \cdot \frac{\mathrm d}{\mathrm d t} \big( i_m(t) \big)+R_m \cdot i_m(t) + V_{EMF}(t)
 \end{equation}
Den modelektromotoriske kraft, $V_{EMF}$ er givet ved ligning \ref{eq:VEMF}.
\begin{equation}
	\label{eq:VEMF}
	V_{EMF}(t) = K_b \cdot \omega(t)
\end{equation}
Med ligning \ref{eq:VEMF} kan ligning \ref{eq:Vm_transient0} omskrives til ligning \ref{eq:Vm_transient1}.
\begin{equation}
	\label{eq:Vm_transient1} 
	V_m(t)=L_m \cdot \frac{\mathrm d}{\mathrm d t} \big( i_m(t) \big)+R_m \cdot i_m(t) +K_b \cdot \omega(t)
 \end{equation}

\subsection{Eksperiment 1}
\subsubsection{Formål}
Bestemmelse af motorens ækvivalente resistans, $R_m$.
\subsubsection{Teori}
Forhindres motoren i at rotere, vil vinkelhastigheden være nul.
Hvis man påfører motoren en DC-spænding $V_m$,
og venter til motorens respons har nået steady-state,
så vil strømmen igennem motoren være konstant.
Her vil ligning \ref{eq:Vm_transient1} kunne omskrives til ligning \ref{eq:resistans_E1}.
\begin{equation}
	\label{eq:resistans_E1} 
	V_m=R_m \cdot i_m
 \end{equation}
Målinger af sammenhørende steady-state værdier for strøm $i_m$ og spænding $V_m$
kan altså bruges til bestemmelse af den ækvivalente resistans $R_m$.
\subsubsection{Fremgangsmåde}
Motoren låses fast vha. en skiftenøgle og påføres en lav spænding.
Efter nogle sekunder, når transientresponsen er væk,
måles spændingen over og strømmen igennem motoren med multimetre.
Værdierne noteres, og forsøget gentages ved andre spændinger.

De anvendte multimetre er af typen TTi 1604.
\subsubsection{Måleresultater}
I tabel \ref{tb:resistans} findes målingerne af strøm og spænding.
\begin{figure}[th!]
	\label{tb:resistans}
	\centering
	%\begin{tabular}{r|r}
%$V_m$ [V]&$i_m$ [A]\\\hline
%0,509&0,094 \\ %0,5093&0,094 \\
%0,883&0,172 \\ %0,8833&0,172
%1,481&0,297 \\ %1,4809&0,297 
%1,940&0,391 \\ %1,9404&0,391
%2,345&0,470 \\ %2,3448&0,470
%2,788&0,555 \\ %2,7879&0,555
%3,182&0,638 \\ %3,1823&0,638
%3,465&0,664 \\ %3,4652&0,664
%3,835&0,746 \\ %3,8346&0,746
%4,067&0,765 \\ % 4,067&0,765
%4,292&0,830 \\ %4,292&0,830 
%\end{tabular}


\begin{tabular}{r|r|r|r|r|r|r|r|r|r|r|r}
$V_m$ [V]&0,509&0,883&1,1481&1,940&2,345&2,788&3,182&3,465&3,835& 4,067&4,292\\\hline
$i_m$ [A]&0,094&0,172&0,297 &0,391&0,470&0,555&0,638&0,664&0,746&0,765&0,830
\end{tabular}
	\captionsetup{type=table}
	\caption[Strøm-spændingskarakteristik for rotationslåst DC-motor ved steady-state]{Strøm-spændingskarakteristik for rotationslåst DC-motor ved steady-state.}
\end{figure}
\subsubsection{Databehandling}
\todo[inline]{Dataplot med tendenslinje.\\Værdi for $R_m$.\\Diskussion.}
\subsubsection{Konklusion}

\subsection{Eksperiment 2}
\subsubsection{Fremgangsmåde og forsøgsopstilling}
\subsubsection{Databehandling}
\subsubsection{Konklusion}


\subsection{Eksperiment 3}
\subsubsection{Fremgangsmåde og forsøgsopstilling}
\subsubsection{Databehandling}
\subsubsection{Konklusion}


\subsection{Eksperiment 4}
\subsubsection{Fremgangsmåde og forsøgsopstilling}
\subsubsection{Databehandling}

\subsubsection{Konklusion}

\subsection{Opsummering af DC motorparameter}