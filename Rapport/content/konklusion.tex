\section{Konklusion}
\label{sec:konklusion}
Et Pan \& Tilt-system er blevet opbygget til nedskydning af en lerdue i English Skeet.
Med udgangspunkt i en matematisk model af systemet er to digitale regulatorer
blevet designet: En PID-regulator til pan-systemet og en PI-regulator til tilt-systemet.

Regulatorerne er implementeret på en mikrocontroller, og modtager positionsfeedback
fra en FPGA gennem SPI.
Reguleringssløjfernes realtidskrav imødekommes ved brug af operativsystemet FreeRTOS.

Efter manuel justering af PID-regulatoren til pan-systemet er det blevet fundet,
at kravene til systemets performance ikke kan imødekommes i alle tilfælde.
I nogle af de udførte tracking-forsøg var systemet dog i stand til at følge lerduens bevægelse
iht. kravspecifikationen,
og i alle forsøgene fulgtes bevægelsen nøjagtigt i min. 0,42 [s] efter lerduen havde nået sit toppunkt.

\section{Perspektivering}
Det vurderes, at mere avancerede regulatortyper er nødvendige til den nøjagtige tracking af lerduen.
\todo[inline,author=Mikael,color=red]{Flere encoderticks!!! Mega fedt!}
\todo[inline,author=Mikael,color=red]{Mere slidstærke drivremme ELLER generelt slidstærk drivmekanisme}
\todo[inline,author=Mikael]{Uddyb det med avancerede regulatortyper! F.eks. adaptive!}
\todo[inline,author=Mikael]{Hvad kan man bruge skidtet til i fremtiden!?\\Snak om mere avancerede regulatortyper.\\Bedre udstyr?\\2. ordens interpolation af parabelpunkter så lerduebevægelsen kunne forudsiges.}
\todo[inline,author=Cirkeline,color=green]{PIZZA!!!}