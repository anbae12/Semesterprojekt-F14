\section{Konklusion}
\label{sec:konklusion}
Et Pan \& Tilt-system er blevet opbygget til nedskydning af en lerdue i English Skeet.
Med udgangspunkt i en matematisk model af systemet er to digitale regulatorer
blevet designet: En PID-regulator til pan-systemet og en PI-regulator til tilt-systemet.

Regulatorerne er implementeret på en mikrocontroller, og modtager positionsfeedback
fra en FPGA gennem SPI.
Reguleringssløjfernes realtidskrav imødekommes ved brug af operativsystemet FreeRTOS.

Efter manuel justering af PID-regulatoren til pan-systemet er det blevet fundet,
at kravene til systemets performance ikke kan imødekommes i alle tilfælde.
I nogle af de udførte tracking-forsøg var systemet dog i stand til at følge lerduens bevægelse
iht. kravspecifikationen,
og i alle forsøgene fulgtes bevægelsen nøjagtigt i min. 0,42 [s] efter lerduen havde nået sit toppunkt.

\section{Videreudvikling}
En højere opløsning på vinklen kunne have forbedret performance og givet mere nøjagtig tracking af lerduen.
Dette kan opnås ved flere Hall sensorer monteret på motorerne, eller ved en anden gearing på PTS.
Hvis fx gearingen ændredes fra 1:3 til 1:6 ville opløsningen blive fordoblet. Samtidig ville responsens tophastighed
blive sænket. Videre arbejde kunne altså undersøge mulighederne for mere nøjagtig tracking af lerduen og deres
indflydelse på performance i tidsdomænet.

Til hurtigere respons på trackingen ville en mere avanceret regulator muligvis kunne anvendes.
Fx. kunne videre arbejde omfatte design af en regulator til det koblede system.
Samtidig ville det være af høj relevans at minimere følsomheden overfor slid i drivremmene,
enten ved at erstatte drivremmene med mere slidstærke varianter, eller ved løbende tilpasning
af regulatoren til parametervariationen i PTS's overføringsfunktioner.
En undersøgelse af andre filtreringsmuligheder for D-leddets filter i PID-regulatoren ville kunne kaste
lys over, om det implementerede D-filter er optimalt til applikationen.
Reguleringen ville kunne forbedres ved forudsigelse af lerduens bevægelse med interpolation
af de samplede koordinater for lerduen.
Videre arbejde med systemet til samme applikation burde udnytte,
at lerduens bevægelse tilnærmelsesvis er parabelformet og dermed forudsigelig.