\section{Mikrocontroller}

%\todo[author=Åse,inline]{Der er et afsnit om SPI. Skal det, der står om SPI i dette afsnit, være i det afsnit?} 

\label{sec:mikrocontroller}
\subsection{Krav til mikrocontrolleren}
\todo[inline, author=Michael]{Husk at TEST sektionen skal se om vi opfylder disse krav. Hvis ikke vi tester tingene skal de slettes fra dette afsnit!!!}
Mikrocontrolleren har følgende opgaver: 
%\todo[inline, author=Michael]{YOU CANT TOUCH THIS - lad mig lige skrive færdig!}
%\todo[inline, author=Michael]{Denne sektion mangler generelt at blive forkortet. }
%\todo[inline, author=Michael]{Anders, kan vi ændre layouttet så listerne ikke tager så meget plads? }


\begin{itemize}
\itemsep1pt
	\item Afvikling af regulatorerne.
	\item Kommunikation med FPGA vha. SPI.
	\item Modtage kommandoer fra PC'en via UART.
	\item Sende relevant data til brugeren via UART.
\end{itemize}

Regulatorerne skal afvikles i hård realtid\footnote{Dvs. at tasken skal afvikles, inden dens deadline - inden for en periode.}, da det ellers er vanskelligt at modellere forsinkelsen matematisk.


SPI\footnote{Serial Peripheral Interface, også kaldet SSI} bruges til at overføre PTS's position fra FPGA til mikrocontroller. Desuden overføres beregnet duty-cycle fra mikrocontroller til FPGA. Kommunikationen skal være tilpas hurtig, så regulatorerne ikke beregner på forældet data. Forsinkelsen for overførsel af data fra begge motorer må ikke overstige 5\% af sampletiden. Ved $T_s = \frac{1}{600}$ er det: 
\begin{equation}
	T_{spi delay} = \frac{1}{600} \cdot 0.05 = 8,33 \cdot 10^{-5}[s]
\end{equation}


Systemet skal kunne modtage brugerinput gennem et terminalprogram. Brugerinterfacet er ikke tidskritisk. Der skal være mulighed for at udlæse væsentlige systemparametre. F.eks. PTS' position, ønsket position, aktuel Pwm duty cycle osv.

\todo[inline, author=Michael]{Måske skal disse systemparametre defineres mere fast.}

%\begin{itemize}
%\itemsep1pt
%	\item PTS' position. 
%	\item Ønsket position. 
%	\item Afvigelse fra ønsket position. 
%\end{itemize}

Desuden skal systemet gemme disse informationer i en log-fil, som kan udlæses efter systemet er færdig med at tracke målet. 

Systemet skal kunne reagere på følgende bruger inputs:


\begin{itemize}
\itemsep1pt
	\item Start tracking
	\item Stop tracking
	\item Reset systemet
	\item Gå til koordinat (x,y,z)
	\item Set PWM
	\item Læs log
\end{itemize}

%%%%%%%%%%%%%%%%%%%%%%%%%%%%%%%%%%%%%%%%%%%%%%%%%%%%%

\subsection{Beskrivelse af valgt(?) mikrocontroller.}
Den udleverede mikrocontroller er en 32 bit Stellaris LM3S6965, baseret på en ARM Cortex M3 kerne. Den opererer ved 50 [MHz] og bruger Thump 2 instruktionssættet. Den indeholder 256 [kB] flash hukommelse samt 64 [kB] SRAM. 
\citep{lm3s6965}

SPI er indbygget som et hardwaremodul, med seperate FIFO buffere til afsendelse og modtagelse. Hardwaren sender data fra den ene den buffer og lægger modtaget data i den anden. Derfor er det ikke nødvendigt for programmøren at holde styr på timingen. Størrelsen af hvert dataframe kan være 4 - 16 bits. 

UART er også implementeret som et hardwaremodul. Modulet indeholder seperate FIFO buffere til afsendelse og modtagelse af data, hver med plads til 16 datawords (med 8 bit som max. størrelse). 



%%%%%%% Den næste sætning er vist bare fyldtekst
%Controlleren indeholder også et væld af hardwaremoduler til andre funktioner, men disse er ikke relevante for projektet.

%%%%%%%%%%%%%%%%%%%%%%%%%%%%%%%%%%%%%%%%%%%%%%%%%%%%%

\subsection{Valg af skedulering/operativsystem}
Der er undersøgt 4 forskellige muligheder for skedulering: 

\begin{itemize}
\itemsep1pt
	\item CoOS\footnote{\citep{www.coocox.com/CoOS.htm}} 
	\item freeRTOS\footnote{\citep{freertos.org}}
	\item Run-To-Complete-Scheduler\footnote{Introduceret i EMP-kurset}
	\item Super-loop
\end{itemize}

De to førstnævnte er fulde operativsystemer med flere mulige skeduleringsalgoritmer. 
Til valget er der lagt vægt på følgende parametre: 

\begin{itemize}
\itemsep1pt
	\item Opfylder kravene til realtidsafvikling af regulatorerne. 
	\item Muligheden for at (gen)bruge elementer fra EMP-kurset.
\end{itemize}

\begin{table}[h!]
\begin{tabular}{|l|l|l|l|l|}
\hline 
 & coOS & freeRTOS & RTCS & Super loop \\ 
\hline 
Skedulering & Preemptive priority  & Preemptive priority  & Non-preemptive & Non-preemptive  \\ 
			& /round robin		&	/round robin & &	\\
\hline 
Queues & Ja & Ja & Nej & Nej \\ 
\hline 
Semaphore & ?  & Counting + binære & Binære  & Nej  \\ 
\hline 
Introduceret & Nej & Ja & Ja & Ja \\ 
i EMP kurset &   &   &   &   \\
\hline 
\end{tabular} 
\caption{Specifikationer for de undersøgte systemer}
\label{tb:os_comparison}
\end{table}
\todo[inline, author=Michael]{Ikke færdig - tilføj noget om hvor lang tid et context switch tager. Dispatch latency.}
%
\subsection{Implementering}
%disp: 
% task diagram
% beskrivelse af tasks
% implementering af enkelt task (CONTROL TASK). 
%   - hvordan sikres at denne kører realtid.
% 
% beskrivelse af interfacet
% 
\todo[inline, author=Michael]{Venter på at kodens struktur er helt færdig - inden da giver det ikke mening at skrive dette afsnit færdigt. }
Valget af freeRTOS som operativsystem gjorde det muligt at opdele programmet i seperate tasks. Taskdiagrammet er vist på figur \ref{fig:task_diagram}. 

\todo[inline, author=Michael]{I figuren burde vi bruge rette linjer istedet for de afrundede. Desuden skal kommunikationen med SPI indtegnes på en eller anden måde. Irrelevante semaphorer over de fleste køer skal også fjernes. }
% TASK DIAGRAM
\begin{figure}[!h]
\centering
\begin{tikzpicture}[node distance = 3.2cm]
	\include*{./graphics/uc_task_diagram}
\end{tikzpicture}
\caption[Task diagram]{Taskdiagrammet viser programmets opdeling i tasks.}
\label{fig:task_diagram}
\end{figure}

\subsubsection{Beskrivelse af de enkelte tasks}
\begin{description}
	\item[Uart send] Tager indholdet fra uart\_ out\_ buffer og ligger dem ind i UART-modulets hardware FIFO buffer. Dette er nødvendigt fordi hw. bufferen kun indeholder 16 pladser. \citep[Side. 430]{lm3s6965}
	\item[Uart receive] Læser UART-modulets hardware buffer og lægger de modtagne data over i uart\_ in \_ buffer. 
	\item[Interface] Tolker på brugerinputs og sender relevante kommandoer til Control- og read\_ pos-tasken.
	\item[read pos] Opdaterer target\_ pos med målets nuværende koordinater 60 gange i sekundet. I vores projekt læser den blot værdierne fra et array, men den kan sagtens udskiftes med en anden task der f.eks. tolker på input fra et kamera.
	\item[Control] Afvikler regulatorerne. Se mere indgående beskrivelse i sektion \ref{sec:control_task}. Denne task afvikles i realtid, med en frekvens på 600 Hz. 
	\item[log updater] Denne task opdaterer logfilen  med data fra "status" køen.
\end{description}


%Semaphorer på target_pos. 
%Ingen semaphorer på køer, undtagen uart out buffer


\subsubsection{Control task}
\label{sec:control_task}

\subsubsection{SPI}
\label{sec:spi-implementering}
Det blev vedtaget at mikrocontrolleren var masteren og FPGA'en blev slaven, da FPGA'en helst skulle være så enkel som mulig.

Mikrocontrolleren har indbygget tre forskellige versioner af SPI. De tre typer er TI Synchronous Serial Frame Format (SSFT), Freescale og Microwire. 
%Til at overfører data mellem Mikrocontrolleren og FPGA’en skal der bruges en enkel protokol med full duplex. \todo[author=Anders]{Hovedgrundene til valget af SPI protokol.}
%Hovedgrundene for at protokollen skal være enkel, er at der ikke sendes meget data, dataene er enkel, og afstanden, den sendes, er kort. 
%Microwire er ikke full duplex. Så det er enten SSFt eller Freescale, der skulle bruges. Da SSFT havde færre opsætningsmuligheder, blev den valgt som protokol. 
%I SSFT ligger svaret på nogle af overvejelserne. Overvejelserne har været om MOSI skal trigger på rising eller falling edge, og hvordan Slave Select skulle opføre sig.
Microwire blev udelukket da den ikke er full-duplex og dermed ikke en ren SPI standard. 
Freescale formatet blev udelukket da opsætningen af denne var mere kompliceret end for SSFT, uden at tilføje ekstra muligheder.

\todo[author=Åse,inline]{kan omfomulere Microwire blev...}

Størrelse af data framen blev valgt til 16 bit ($2^4$), da det giver mulighed for at sende motorens hastighed og deres position (begge 11 bit) samt kontroldata. Indholdet af de enkelte pakker/datawords kan ses på figur \label{uc:protokol1}.

\begin{figure}[th!]
\centering
\begin{tabular}{c|c|c|c|c}
1 motorbit &1 retningsbit & 1 set PWMbit & 2 ignored bits & 11 PWM dutycyclebits\\
\end{tabular}
 \begin{tabular}{c|c|c}
 1 motorbit & 4 ignored bits & 11 decoderbits
 \end{tabular}
\captionsetup{type=figure}
\caption[SPI framestruktur]{Øverst ses framestrukturen fra mikrocontrolleren til FPGA'en. Nederst ses framestrukturen fra FPGA'en til mikrocontrolleren.}
\label{uc:protokol1}
\end{figure}

\todo[author=Åse,inline]{Skriv en figur tekst senere} 

\todo[inline, author=Michael]{Alternativ version. Please vote :)}

\begin{figure}[h!]
\centering
\subfloat[Mikrocontroller til FPGA.\label{fig:spi_to_fpga}]{%
	\begin{tabular}{|c|}
    \hline 
    Motor (1 bit) \\ 
    \hline 
    Retning (1 bit) \\ 
    \hline 
    Set PWM (1 bit) \\ 
    \hline 
    N/A (2 bit) \\ 
     \\ 
    \hline 
     \\ \\ \\ \\ 
    PWM duty cycle \\ 
    (11 bits) \\ \\ \\ \\ \\ \\ 
    \hline 
  \end{tabular} 
}
\qquad
\subfloat[FPGA til Mikrocontroller.\label{fig:spi_from_fpga}]{%
	\begin{tabular}{|c|}
    \hline 
    Motor (1 bit) \\ 
    \hline \\
    N/A (4 bit) \\ 
     \\ \\
    \hline 
     \\ \\ \\ \\ 
    PWM duty cycle \\ 
    (11 bits) \\ \\ \\ \\ \\ \\ 
    \hline 
  \end{tabular} 
}
\caption[Lerduens parabel i 2D]{Viser parablen af lerduens bane i 2D.}
\end{figure}





\subsection{Test} 
% Test af SPI overførsel
% Test af realtidsafvikling af reguleringssløjferne, 
% lever vi op til kravene?
% 





\subsection{Delkonklusion}
