\section{Mikrokontroller}
\label{sec:mikrokontroller}
\subsection{Krav til mikrokontrolleren}
Mikrokontrolleren har følgende opgaver: 
\todo[inline, author=Michael]{YOU CANT TOUCH THIS - lad mig lige skrive færdig!}
\todo[inline, author=Michael]{}
\begin{itemize}
	\item Afvikling af regulatorerne.
	\item Kommunikation med FPGA vha. SPI.
	\item Modtage kommandoer fra PC'en via UART.
	\item Sende relevant data til brugeren via UART.
\end{itemize}

Afvikling af regulatorerne skal afvikles i realtid, da det ellers er vanskelligt at modellere forsinkelsen. 


SPI\footnote{Serial Peripheral Interface, også kaldet SSI} bruges til at overføre PTS's position fra FPGA til mikrokontroller. Desuden overføres beregnet duty-cycle fra mikrokontroller til FPGA. Kommunikationen skal være tilpas hurtig, så regulatorerne ikke beregner på forældet data. 
\todo[inline, author=Michael]{Overførslen af et dataword svarer til sampling tiden. Tiden for overførsel af et data-word skal derfor være signifikant lavere end sampletiden.  }

Systemet skal kunne modtage bruger inputs gennem et terminalprogram. Brugerinterfacet er ikke tidskritisk. Der skal være mulighed for at udlæse systemparametre:

\begin{itemize}
	\item PTS' position. 
	\item Ønsket position. 
	\item Afvigelse fra ønsket position. 
\end{itemize}

Desuden skal systemet gemme disse informationer i en log-fil, som kan udlæses efter systemet er færdig med at tracke målet. 

Systemet skal kunne reagere på følgende bruger inputs:


\begin{itemize}
	\item Start tracking
	\item Stop tracking
	\item Reset systemet
	\item Gå til koordinat (x,y,z)
\end{itemize}


\subsection{Valg af skedulering}
Der er undersøgt 4 forskellige mulige skeduleringsalgoritmer: 

\begin{itemize}
	\item CoOS\footnote{coocox.com/CoOS.htm}
	\item freeRTOS\footnote{freertos.org}
	\item Run-To-Complete-Scheduler\footnote{Introduceret i EMP-kurset}
	\item Super-loop
\end{itemize}

Til valget er der lagt vægt på følgende parametre: 

\begin{itemize}
	\item Opfylder kravene til realtidsafvikling af regulatorerne. 
	\item Muligheden for at (gen)bruge elementer fra EMP-kurset.
\end{itemize}

\begin{table}[h!]
\begin{tabular}{|l|l|l|l|l|}
\hline 
 & coOS & freeRTOS & RTCS & Super loop \\ 
\hline 
Skedulering & Preemptive priority  & Preemptive priority  & Non-preemptive & Non-preemptive  \\ 
			& /round robin		&	/round robin & &	\\
\hline 
Queues & Ja & Ja & Nej & Nej \\ 
\hline 
Semaphore & ?  & Counting + binære & Binære  & Nej  \\ 
\hline 
• & • & • & • & • \\ 
\hline 
\end{tabular} 
\caption{Specifikationer for de undersøgte systemer}
\label{tb:os_comparison}
\end{table}

\subsection{Implementering}
\subsection{Test} 
\subsection{Delkonklusion}

