\section{Mikrocontroller}
\label{sec:mikrocontroller}
%
\subsection{Krav til mikrocontrolleren}
\todo[inline, author=Michael]{Husk at TEST sektionen skal se om vi opfylder disse krav. Hvis ikke vi tester tingene skal de slettes fra dette afsnit!!!}
Mikrocontrolleren har følgende opgaver: 
%\todo[inline, author=Michael]{YOU CANT TOUCH THIS - lad mig lige skrive færdig!}
%\todo[inline, author=Michael]{Denne sektion mangler generelt at blive forkortet. }
%\todo[inline, author=Michael]{Anders, kan vi ændre layouttet så listerne ikke tager så meget plads? }


\begin{itemize}
\itemsep1pt
	\item Afvikling af regulatorerne.
	\item Kommunikation med FPGA vha. SPI.
	\item Modtage kommandoer fra PC'en via UART.
	\item Sende relevant data til brugeren via UART.
\end{itemize}

Regulatorerne skal afvikles i hård realtid\footnote{Dvs. at tasken skal afvikles, inden dens deadline - inden for en periode.}, da det ellers er vanskelligt at modellere forsinkelsen matematisk.


SPI\footnote{Serial Peripheral Interface, også kaldet SSI} bruges til at overføre PTS's position fra FPGA til mikrocontroller. Desuden overføres beregnet duty-cycle fra mikrocontroller til FPGA. Kommunikationen skal være tilpas hurtig, så regulatorerne ikke beregner på forældet data. Forsinkelsen for overførsel af data fra begge motorer må ikke overstige 5\% af sampletiden. Ved $T_s = \frac{1}{600}$ er det: 
\begin{equation}
	T_{spi delay} = \frac{1}{600} \cdot 0.05 = 8,33 \cdot 10^{-5}[s]
\end{equation}


Systemet skal kunne modtage brugerinput gennem et terminalprogram. Brugerinterfacet er ikke tidskritisk. Der skal være mulighed for at udlæse væsentlige systemparametre såsom: PTS' position, ønsket position, aktuel PWM duty cycle osv.

\todo[inline, author=Michael]{Måske skal disse systemparametre defineres mere fast.}

%\begin{itemize}
%\itemsep1pt
%	\item PTS' position. 
%	\item Ønsket position. 
%	\item Afvigelse fra ønsket position. 
%\end{itemize}

Desuden skal systemet gemme disse informationer i en log-fil, som kan udlæses efter systemet er færdig med at tracke målet. 

%%%%%%%%%%%%%%%%%%%%%%%%%%%%%%%%%%%%%%%%%%%%%%%%%%%%%

\subsection{Beskrivelse af valgt(?) mikrocontroller.}
Den udleverede mikrocontroller er en 32 bit Stellaris LM3S6965, baseret på en ARM Cortex M3 kerne. Den opererer ved 50 [MHz] og bruger Thump 2 instruktionssættet. Den indeholder 256 [kB] flash hukommelse samt 64 [kB] SRAM. 
\citep{lm3s6965}

SPI er indbygget som et hardwaremodul, med seperate FIFO buffere til afsendelse og modtagelse. Hardwaren sender data fra den ene den buffer og lægger modtaget data i den anden. Derfor er det ikke nødvendigt for programmøren at holde styr på timingen. Størrelsen af hvert dataframe kan være 4 - 16 bits. 

UART er også implementeret som et hardwaremodul. Modulet indeholder seperate FIFO buffere til afsendelse og modtagelse af data, hver med plads til 16 datawords (på max. 8 bit). 


%%%%%%%%%%%%%%%%%%%%%%%%%%%%%%%%%%%%%%%%%%%%%%%%%%%%%

\subsection{Valg af schedulering/operativsystem}
Der er undersøgt 4 forskellige muligheder for schedulering. CoOS\footnote{\citep{www.coocox.com/CoOS.htm}} og freeRTOS\footnote{\citep{freertos.org}} er komplette real-tids operativsystemer. 
Run-To-Complete-Scheduler er en simpel scheduler introduceret i EMP kurset. Ved brug af Super-loop metoden afvikles al koden i main-loopet vha. ticks.
De fire mulige scheduleringer er beskrevet i tabel \ref{tb:os_comparison}.

%\begin{itemize}
%\itemsep-2pt
%	\item Opfylder kravene til realtidsafvikling af regulatorerne. 
%	\item Muligheden for at (gen)bruge elementer fra EMP-kurset.
%\end{itemize}

\begin{table}[h!]
\begin{tabular}{|l|l|l|l|l|}
\hline 
 & coOS & freeRTOS & RTCS & Super loop \\ 
\hline 
Schedulering & Preemptive priority  & Preemptive priority  & Non-preemptive & Non-preemptive  \\ 
			& /round robin		&	/round robin & &	\\
\hline 
Queues & Ja & Ja & Nej & Nej \\ 
\hline 
Semaphore & Counting og Binære  & Counting og Binære & Binære  & Nej  \\ 
\hline 
Introduceret & Nej & Ja & Ja & Ja \\ 
i EMP kurset &   &   &   &   \\
\hline 
\end{tabular} 
\caption{Specifikationer for de undersøgte systemer}
\label{tb:os_comparison}
\end{table}

Til valget er der lagt vægt på om scheduleringen opfylder kravet til realtidsafvikling af regulatorerne, samt muligheden for at genanvende elementer fra EMP-kurset.

\todo[inline, author=Michael]{Ikke færdig - tilføj noget om hvor lang tid et context switch tager. Dispatch latency.}
%
\subsection{Implementering}
%disp: 
% task diagram
% beskrivelse af tasks
% implementering af enkelt task (CONTROL TASK). 
%   - hvordan sikres at denne kører realtid.
% 
% beskrivelse af interfacet
% 
\todo[inline, author=Michael]{Venter på at kodens struktur er helt færdig - inden da giver det ikke mening at skrive dette afsnit færdigt. }

Valget af freeRTOS som operativsystem gjorde det muligt at opdele programmet i seperate tasks. Taskdiagrammet er vist på figur \ref{fig:task_diagram}. 

% TASK DIAGRAM
\begin{figure}[!h]
\centering
\begin{tikzpicture}[node distance = 3.2cm]
	\include*{./graphics/uc_task_diagram}
\end{tikzpicture}
\caption[Task diagram]{Taskdiagrammet viser programmets opdeling i tasks.}
\label{fig:task_diagram}
\end{figure}

\subsubsection{Beskrivelse af de enkelte tasks}
\begin{description}
\itemsep-3pt
	\item[UART] Grundet UART hw. bufferen kun indeholder 16 pladser \citep[Side. 430]{lm3s6965}. Sørger en Send og Recieve task for at sende data videre. Ekstra Semaphore til at sikre kun en task kan skrive til Uart af gangen er implementeret.
	\item[Interface] Tolker på brugerinputs og sender relevante kommandoer til Control tasken.
	\item[Control] Afvikler regulatorerne. Se mere indgående beskrivelse i sektion \ref{sec:control_task}. Denne task afvikles i realtid, med en frekvens på 600 Hz. 
	Denne task sørger for at SPI Kommunikationen med FPGA'en bliver afviklet på de rigtige tidspunkter.
	\item[Logger] Denne task opdaterer logfilen med data fra "status" køen.
\end{description}

\subsubsection{Interface task}
\label{sec:interface}
Bruger Interfacet gør det overskueligt at styre og skifte mellem de forskellige tilstande som systemet kan køres i.
Der er lavet funktioner som bruges til test af specifikke funktioner og funktioner som starter en større test.
Formålet er at gøre det simpelt for gruppen, det er altså ikke tænkt til en kunde som ikke ved noget om systemet.
Fjernelse af implementerede funktioner for at gøre det simpelt har ikke været en prioritet. 
De primære funktioner er at sætte en position eller starte tracking af lerduen.
Hvis det indtastede ikke bliver genkendt bliver en liste over gyldige funktioner udskrevet.

\begin{table}[h!]
\begin{tabular}{|l|l|l|}
\hline 
Kommando: & Funktion: & Parametre: \\ 
\hline 
read & Slå log til eller fra.  &   \\ 
\hline 
start & Start tracking af lerdue &   \\ 
\hline 
stop & Stop systemet &   \\ 
\hline 
open & Start åben sløjfe test &   \\ 
\hline 
reset & Gå til udgangsposition &   \\ 
\hline 
C xxx.yyy.zzz & Gå til  kartesisk koord. (x,y,z) & x,y,z = Koordinaterne i meter \\ 
\hline 
A aaaa.bbbb & Gå til sfærisk koord.  & a = Tilt, b = Pan, begge i ticks. \\ 
\hline 
Pmfxx & Sæt dutycycle for motor m & m = Motor ("A" ell. "B"), \\ 
  & & f = fortegn ($+$ ell. $-$),  \\
  & & xx = PWM dutycycle i \%\\
\hline 
\end{tabular} 
\caption{Viser UART-interfacets kommandoer.}
\label{tb:uart_interface}
\end{table}



\subsubsection{Control task}
\label{sec:control_task}

Control tasken henter et koordinat, transformerer koordinatet udregner en PWM med PID controlleren og opdaterer loggen.
hver gang control tasken bliver kørt bliver der hentet en position ind over SPI og sendt ny PWM. 
Control tasken ved hvilke ting som er relevant at gemme i loggen, så ved start af tracking bliver positioner og pwm gemt i loggen.
I kodeudsnittet \ref{ctrl_task} ses der tasken som den bliver kørt hvis den er i tracking stadiet.

\lstinputlisting[language=c, 
        firstnumber=120,
        firstline=120, 
        lastline=122,
        ]{codeexample/ctrl_task.c}
\(\vdots\)
\lstinputlisting[language=c, 
        firstnumber=155,
        firstline=155, 
        lastline=163,
        ]{codeexample/ctrl_task.c}
\(\vdots\)
\lstinputlisting[language=c, 
        firstnumber=180,
        firstline=180, 
        lastline=186,
        caption={ af ctrl\_task.c.\label{ctrl_task}}
        ]{codeexample/ctrl_task.c}
        
\subsubsection{SPI}
\label{sec:spi-implementering}
Det blev vedtaget at mikrocontrolleren var masteren og FPGA'en blev slaven, da FPGA'en helst skulle være så enkel som mulig.

Mikrocontrollerens SPI modul har indbygget tre forskellige standarder: TI Synchronous Serial Frame Format (SSFT), Freescale og Microwire. 
%Til at overfører data mellem Mikrocontrolleren og FPGA’en skal der bruges en enkel protokol med full duplex. \todo[author=Anders]{Hovedgrundene til valget af SPI protokol.}
%Hovedgrundene for at protokollen skal være enkel, er at der ikke sendes meget data, dataene er enkel, og afstanden, den sendes, er kort. 
%Microwire er ikke full duplex. Så det er enten SSFt eller Freescale, der skulle bruges. Da SSFT havde færre opsætningsmuligheder, blev den valgt som protokol. 
%I SSFT ligger svaret på nogle af overvejelserne. Overvejelserne har været om MOSI skal trigger på rising eller falling edge, og hvordan Slave Select skulle opføre sig.
Microwire blev udelukket da den ikke er full-duplex og dermed ikke en SPI standard. 
Freescale formatet blev udelukket da opsætningen af denne var mere kompliceret end for SSFT, uden at tilføje ekstra muligheder.

Størrelse af data framen blev valgt til 16 bit, hvilket giver mulighed for at sende motorens hastighed og deres position (begge 11 bit) samt kontroldata. Opbygningen af de to datawords kan ses på figur \ref{fig:protokol1}.

\begin{figure}[h!]
\centering
\subfloat[Mikrocontroller til FPGA.\label{fig:spi_to_fpga}]{%
\begin{tabular}{c|c|c|c|c}
Motor (1 bit) &Retning (1 bit) & Ignore PWM (1 bit)& N/A (2 bit) & PWM duty cycle (11 bit)
\end{tabular}
}
\qquad
\subfloat[FPGA til Mikrocontroller. Der skiftes mellem at sende position for pan og tilt hver gang.
\label{fig:spi_from_fpga}]{%
 \begin{tabular}{c|c|c}
 Motor (1 bit) & N/A (4 bit)  & PWM duty cycle (11 bit)
 \end{tabular}
}
\caption[Indholdet af SPI datawords]{Viser indholdet af SPI datawords. }
\label{fig:protokol1}
\end{figure}


%\begin{figure}[h!]
%\centering
%\subfloat[Mikrocontroller til FPGA.\label{fig:spi_to_fpga}]{%
%\begin{tabular}{|p{4cm}|}
 %   \hline
 %   Motor (1 bit) \\
 %   \hline
 %   Retning (1 bit) \\
 %   \hline
%    Ignore PWM (1 bit) \\
%    \hline
%    N/A (2 bit) \\
 %    \\
  %  \hline
 %    \\ \\ \\ \\
 %   PWM duty cycle \\
 %   (11 bits) \\ \\ \\ \\ \\
 %   \hline
%  \end{tabular}
%}
%\qquad
%\subfloat[FPGA til Mikrocontroller. Der skiftes mellem at sende position for pan og tilt hver gang.
%\label{fig:spi_from_fpga}]{%
%\begin{tabular}{|p{4cm}|}
 %   \hline
  %  Motor (1 bit) \\
%    \hline \\
%    N/A (4 bit) \\
%     \\ \\
%    \hline
%     \\ \\ \\ \\
%    Position (ticks) \\
%    (11 bits) \\ \\ \\ \\ \\
%    \hline
%  \end{tabular}
%}
%\caption[Indholdet af SPI datawords]{Viser indholdet af SPI datawords. }
%\label{fig:protokol1}
%\end{figure}
\todo[inline, author=Anders]{Todo er rettet, men jeg synes personligt den anden noteringsformat er pænere. Det sere "lettere" ud og fylder mindre.}


Figur \ref{fig:spi_to_fpga} viser indholdet af datawords der bliver sendt fra Mikrocontrolleren. 
Motorbittet vælger aktuel motor, retningsbittet vælger hvilken retning denne skal dreje og PWM duty cycle indeholder den ønskede PWM. 
Hvis "Ignore PWM" flaget er sat ignoreres den modtagne besked, mens den aktuelle motorposition returneres stadig som vist på fig.  \ref{fig:spi_from_fpga}. 
%Da SPI er full duplex sendes data fra  Mikrocontrolleren og FPGA’en synkront - FPGA’en sender et dataword samtidig med mikrocontrolleren

 %Mikrocontrolleren sender den  PWM, der skal sættes på den bestemte motor i den bestemte retning. Det er det motor- og retningsbittene bestemmer.


%På Mikrocontrolleren ville man gerne kunne få en position uden, at den sætter en ny PWM. 
%Derfor skal der bruges en bit, hvor FGPA’en kan se om den skal sætte PWM’en, eller om den kun skal sende information tilbage. Der er det PWMset bruges til.

%På figur \ref{fig:spi_from_fpga} ses dataword'et som FPGA'en sender. FPGA'en sender positionen på den bestemte motor. 


\subsection{Test} 
% Test af SPI overførsel
% Test af realtidsafvikling af reguleringssløjferne, 
% lever vi op til kravene?
% 





\subsection{Delkonklusion}
