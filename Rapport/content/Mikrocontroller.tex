\section{Mikrocontroller}
\label{sec:mikrocontroller}
%
\subsection{Krav til mikrocontrolleren}
Mikrocontrolleren har følgende opgaver: 

\begin{itemize}
\itemsep1pt
	\item Afvikling af regulatorerne.
	\item Kommunikation med FPGA vha. SPI.
	\item Modtage kommandoer fra brugeren via UART.
	\item Sende relevant data til brugeren via UART.
\end{itemize}

Regulatorerne skal afvikles i hård realtid\footnote{Dvs. at task'en skal afvikles, inden dens deadline - inden for en samplingperiode \citep{operating_systems_concepts_ed8}.},
da de er designet under antagelsen om periodisk afvikling.

SPI\footnote{Serial Peripheral Interface, også kaldet SSI. Se appendiks \ref{app:spi}.} bruges til at overføre PTS's position fra FPGA til mikrocontroller.
Desuden overføres beregnet duty cycle fra mikrocontroller til FPGA.
Kommunikationen skal være tilpas hurtig, så regulatorerne ikke regner på forældet data.
Forsinkelsen for overførsel af data for begge motorer vurderes til ikke at måtte overstige 5\% af samplingperioden.
Ved $T_s = \frac{1}{600}$ er det: 
\begin{equation}
	T_{\text{SPI delay}} \leq \frac{1}{600} \cdot 5\% = 83,3~[\micro s]
	\label{eq:uc:spi-krav}
\end{equation}

Systemet skal kunne modtage brugerinput gennem et terminalprogram.
Brugerinterfacet er ikke tidskritisk.
Der skal være mulighed for at udlæse væsentlige systemparametre såsom:
PTS's position, ønsket position, aktuel PWM-duty cycle osv.

Desuden skal systemet gemme disse informationer i en logfil,
som udskrives løbende.

%%%%%%%%%%%%%%%%%%%%%%%%%%%%%%%%%%%%%%%%%%%%%%%%%%%%%
\subsection{Beskrivelse af mikrocontroller}
Den udleverede mikrocontroller er en 32-bits Stellaris LM3S6965, 
baseret på en ARM Cortex M3-kerne. 
Den opererer ved 50 [MHz] og bruger Thump 2-instruktionssættet. 
Den indeholder 256 [kB] flash hukommelse samt 64 [kB] SRAM \citep{lm3s6965}.

SPI er indbygget som et hardwaremodul, 
med seperate FIFO-buffere til afsendelse og modtagelse. 
Hardwaren sender data fra den ene FIFO-buffer og 
lægger modtaget data i den anden. 
Derfor er det ikke nødvendigt for programmøren at holde styr på timingen. 
Størrelsen af hvert dataframe kan være 4 - 16 bits. 

UART er også implementeret som et hardwaremodul. 
Modulet indeholder seperate FIFO-buffere til afsendelse og modtagelse af data, hver med plads til 16 datawords à 8 bit. 

%%%%%%%%%%%%%%%%%%%%%%%%%%%%%%%%%%%%%%%%%%%%%%%%%%%%%
\subsection{Valg af operativsystem og schedulering}
Det er valgt at bruge FreeRTOS, da dette styresystem gør brug af tvungen schedulering med prioritering (preemptive priority), hvilket sikrer hård realtid.
Desuden har gruppen erfaring med styresystemet.

\subsection{Implementering}
Systemet er opdelt i separate tasks.
Taskdiagrammet er vist på figur \ref{fig:task_diagram}. 

% TASK DIAGRAM
\begin{figure}[!h]
\centering
\begin{tikzpicture}[node distance = 3.2cm]
	\include*{./graphics/uc_task_diagram}
\end{tikzpicture}
\caption[Taskdiagram]{Taskdiagrammet viser programmets opdeling i tasks.
	Tasks: UART send, UART receive, Interface, Logger og Control.
	Køer: uart out buffer, uart in buffer, control message og status.
	State variabel: log file.}
\label{fig:task_diagram}
\end{figure}

\subsubsection{Beskrivelse af de enkelte tasks}
	\textbf{UART send og -receive}\\ 
	Da UART hardwarebufferen kun indeholder 16 pladser \citep[s. 430]{lm3s6965},
	sørger en Send og en Recieve task for at sende data videre til og fra resten af programmet.
	De to ekstra semaforer som Interface tilgår er til at reservere de to buffere så indtastede og
	printede tekststrenge ikke kan afbrydes (kun én task kan skrive til eller læse fra UART ad gangen).\\
	\textbf{Control}\\ 
	Denne task afvikler regulatorerne. Se mere indgående beskrivelse i sektion \ref{sec:control_task}.
	Denne task afvikles med højeste prioritet, hvilket sikrer realtid med en frekvens på 600 [Hz]. 
	Denne task sørger for at SPI-kommunikationen med FPGA'en bliver afviklet på de rigtige tidspunkter.\\
	\textbf{Logger}\\ 
	Denne task opdaterer logfilen med data fra statuskøen.\\
	\textbf{Interface}\\ 
	Denne task tolker på brugerinputs og sender relevante kommandoer til Control task'en.

\subsubsection{Control task}
\label{sec:control_task}
Control task'en henter en koordinat, transformerer koordinaten,
udregner en PWM-duty cycle for hver regulator og opdaterer loggen.
Hver gang control task'en bliver kørt bliver der hentet to positioner
ind via SPI mens de to nye PWM-duty cycles sendes.
Ved tracking skriver Control task'en positioner og PWM-duty cycles i loggen.

\subsubsection{SPI}
\label{sec:spi-implementering}
Mikrocontrolleren fungerer som master, da den står for systemets timing. FPGA'en er derfor programmeret som SPI slave.

Mikrocontrollerens SPI-modul har indbygget tre forskellige standarder: TI Synchronous Serial Frame Format (SSFT), Freescale og Microwire. 
Microwire blev udelukket da den ikke er full-duplex og dermed ikke en SPI standard. 
Freescale formatet blev udelukket da opsætningen af denne vurderedes mere kompliceret end for SSFT,
uden at tilføje ekstra muligheder.
Valget faldt derfor på SSFT.

Størrelse af dataframen blev valgt til 16 bit, hvilket giver mulighed for at sende
motorens PWM-duty cycle og deres position (begge 11 bit) samt kontroldata.
Opbygningen af de to datawords kan ses på figur \ref{fig:protokol1}.

\begin{figure}[h!]
\centering
\subfloat[Kommando fra mikrocontroller til FPGA.\label{fig:spi_to_fpga}]{%
\begin{tikzpicture}[scale=1]
\include*{./graphics/SPImasterslave}
\end{tikzpicture}
}
\qquad
\subfloat[Svar fra FPGA til mikrocontroller. Der skiftes mellem at sende position for pan og tilt hver gang.
\label{fig:spi_from_fpga}]{%
\begin{tikzpicture}[scale=1]
\include*{./graphics/SPIslavemaster}
\end{tikzpicture}
}
\caption[Indholdet af SPI datawords]{Viser indholdet af SPI datawords. }
\label{fig:protokol1}
\end{figure}

Figur \ref{fig:spi_to_fpga} viser indholdet af datawords der bliver sendt fra mikrocontrolleren. 
Motorbit vælger aktuel motor, retningsbit vælger hvilken retning
denne skal dreje og PWM-duty cycle indeholder den ønskede PWM-duty cycle. 
Hvis setPWM bit ikke er sat ignoreres den modtagne besked,
mens den aktuelle motorposition stadig returneres som vist på figur \ref{fig:spi_from_fpga}. 

Baudrate skal vælges efter kravet om, at to overførsler maks. må tage 83,3 [\micro s], jævnfør 
ligning \ref{eq:uc:spi-krav}. To overførsler svarer til 34 clockcycles ved 
single transfer ved den valgte SPI opsætning. Derfor kan den maksimale værdi for 
clockperioden  $T_C$ ses i ligning \ref{eq:uc:clock-krav}.

\begin{equation}
	T_{\text{C}} \leq \frac{83,3~[\micro s]}{34} = 2,45~[\micro s]
	\label{eq:uc:clock-krav}
\end{equation}

Baudrate vælges til 1 [MHz], så ligning \ref{eq:uc:clock-krav} opfyldes. En 
højere baudrate ville overholde kravet, men gør signalet mere følsomt overfor støj.

\subsection{Test} 
Det testes hvorvidt SPI-overførslen og Control-task'en overholder timingkravene. 
Testen udføres ved at sætte en digital udgang høj mens task'en afvikles, og samtidig måle denne udgang med et oscilloskop. 
Appendiks \ref{sec:uctestappendix} forklarer fremgangsmåden og diskuterer måleresultaterne. 


Det blev fundet at SPI-overførslen af 2 datawords blev udført 40,54 [\micro s], hvilket opfylder kravet på 83,3 [\micro s]. 
Control-task'en blev afviklet periodisk med en afvigelse på under 1\%. 


\subsection{Delkonklusion}
Det blev valgt at benytte den udleverede LM3S6965 mikrocontroller.
Denne understøtter SPI og UART som hardwaremoduler. 
SPI blev implementeret som TI Synchronous Serial Frame Format med en baudrate på 
1 [MHz].

Det blev valgt at bruge FreeRTOS til scheduleringen,
da tvungen schedulering med prioritering sikrer hård realtid. 

Programmet blev delt op i fem task's: UART send, UART receive, Interface, Logger og Control. 
% Al realtidsfunktionalitet blev implementeret i Control-task'en. 
Det var påkrævet at Control-task'en kunne afvikles med en frekvens på 600 [Hz], og den blev derfor tildelt højeste prioritet. 
Task'ens afviklingstid blev målt til max. 360 [\(\micro\)s], hvilket var væsentlig lavere end den max. tilladte tid på \(\frac{1}{600}\) [s]. 
SPI-kommunikationen kunne afvikles på 40,54 [\(\micro\)s], hvilket opfyldte kravet på max. 5\% af Control-task'ens periodetid. 

%Reguleringssløjferne kan ramme inden for kravene i kravspecifikationen.
