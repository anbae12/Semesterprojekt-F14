\section{Mikrokontroller}
\label{sec:mikrokontroller}
\subsection{Krav til mikrokontrolleren}
\todo[inline, author=Michael]{Husk at TEST sektionen skal se om vi opfylder disse krav. Hvis ikke vi tester tingene skal de slettes fra dette afsnit!!!}
Mikrokontrolleren har følgende opgaver: 
\todo[inline, author=Michael]{YOU CANT TOUCH THIS - lad mig lige skrive færdig!}
\todo[inline, author=Michael]{Denne sektion mangler generelt at blive forkortet. }
\todo[inline, author=Michael]{Anders, kan vi ændre layouttet så listerne ikke tager så meget plads? }
\begin{itemize}
	\item Afvikling af regulatorerne.
	\item Kommunikation med FPGA vha. SPI.
	\item Modtage kommandoer fra PC'en via UART.
	\item Sende relevant data til brugeren via UART.
\end{itemize}

Regulatorerne skal afvikles i realtid, da det ellers er vanskelligt at modellere forsinkelsen matematisk.


SPI\footnote{Serial Peripheral Interface, også kaldet SSI} bruges til at overføre PTS's position fra FPGA til mikrokontroller. Desuden overføres beregnet duty-cycle fra mikrokontroller til FPGA. Kommunikationen skal være tilpas hurtig, så regulatorerne ikke beregner på forældet data. Forsinkelsen for overførsel af data må ikke overstige 5\% af sampletiden. 
\todo[inline, author=Michael]{Ved Ts = 1/600 er det $8,33 * 10^5$. Indsæt de endelige tal når matematikgruppen når videre med tallene. }

Systemet skal kunne modtage brugerinput gennem et terminalprogram. Brugerinterfacet er ikke tidskritisk. Der skal være mulighed for at udlæse systemparametre:

\begin{itemize}
	\item PTS' position. 
	\item Ønsket position. 
	\item Afvigelse fra ønsket position. 
\end{itemize}

Desuden skal systemet gemme disse informationer i en log-fil, som kan udlæses efter systemet er færdig med at tracke målet. 

Systemet skal kunne reagere på følgende bruger inputs:


\begin{itemize}
	\item Start tracking
	\item Stop tracking
	\item Reset systemet
	\item Gå til koordinat (x,y,z)
	\item Set PWM
	\item Læs log
\end{itemize}

%%%%%%%%%%%%%%%%%%%%%%%%%%%%%%%%%%%%%%%%%%%%%%%%%%%%%

\subsection{Beskrivelse af valgt(?) mikrocontroller.}
\todo[inline, author=Michael]{Tilføj datasheet til kildelisten}
Den udleverede mikrocontroller er en 32 bit Stellaris LM3S6965, baseret på en ARM Cortex M3 kerne. Den opererer ved 50 mHz og bruger Thump 2 instruktionssættet. Den indeholder 256 KB flash hukommelse samt 64 KB SRAM. 

SPI er indbygget som et hardwaremodul, med seperate FIFO buffere til afsendelse og modtagelse. Hardwaren sender data fra den ene den buffer og lægger modtaget data i den anden. Derfor er det ikke nødvendigt for programmøren at holde styr på timingen. Størrelsen af hvert dataframe kan være 4 - 16 bits. 

UART er også implementeret som et hardwaremodul. Modulet indeholder seperate FIFO buffere til afsendelse og modtagelse af data, hver med plads til 16 datawords (max 8 bit). 

Kontrolleren indeholder også et væld af hardwaremoduler til andre funktioner, men disse er ikke relevante for projektet.



%%%%%%%%%%%%%%%%%%%%%%%%%%%%%%%%%%%%%%%%%%%%%%%%%%%%%

\subsection{Valg af skedulering}
Der er undersøgt 4 forskellige mulige skeduleringsalgoritmer: 

\begin{itemize}
	\item CoOS\footnote{coocox.com/CoOS.htm}
	\item freeRTOS\footnote{freertos.org}
	\item Run-To-Complete-Scheduler\footnote{Introduceret i EMP-kurset}
	\item Super-loop
\end{itemize}

Til valget er der lagt vægt på følgende parametre: 

\begin{itemize}
	\item Opfylder kravene til realtidsafvikling af regulatorerne. 
	\item Muligheden for at (gen)bruge elementer fra EMP-kurset.
\end{itemize}

\begin{table}[h!]
\begin{tabular}{|l|l|l|l|l|}
\hline 
 & coOS & freeRTOS & RTCS & Super loop \\ 
\hline 
Skedulering & Preemptive priority  & Preemptive priority  & Non-preemptive & Non-preemptive  \\ 
			& /round robin		&	/round robin & &	\\
\hline 
Queues & Ja & Ja & Nej & Nej \\ 
\hline 
Semaphore & ?  & Counting + binære & Binære  & Nej  \\ 
\hline 
• & • & • & • & • \\ 
\hline 
\end{tabular} 
\caption{Specifikationer for de undersøgte systemer}
\label{tb:os_comparison}
\end{table}
\todo[inline, author=Michael]{Ikke færdig}
\subsection{Implementering}
%disp: 
% task diagram
% beskrivelse af tasks
% implementering af enkelt task (CONTROL TASK). 
%   - hvordan sikres at denne kører realtid.
% 
% beskrivelse af interfacet
% 

Valget af freeRTOS som operativsystem gjorde det muligt at opdele programmet i seperate tasks. Taskdiagrammet er vist på figur \ref{fig:task_diagram}. 

\todo[inline, author=Michael]{I figuren burde vi bruge rette linjer istedet for de afrundede. Desuden skal kommunikationen med SPI indtegnes på en eller anden måde. Irrelevante semaphorer over de fleste køer skal også fjernes. }
% TASK DIAGRAM
\begin{figure}[!h]
\centering
\begin{tikzpicture}[node distance = 3.2cm]
	\include*{./graphics/uc_task_diagram}
\end{tikzpicture}
\caption[Task diagram]{Taskdiagrammet viser programmets opdeling i tasks.}
\label{fig:task_diagram}
\end{figure}

\subsubsection{Beskrivelse af de enkelte tasks}
\begin{description}
	\item[Uart send] Tager indholdet fra uart\_ out\_ buffer og ligger dem ind i UART-modulets hardware FIFO buffer. Dette er nødvendigt fordi hw. bufferen kun indeholder 16 pladser\footnote{Jævnfør s. 430 i LM3S6965}.
	\item[Uart receive] Læser UART-modulets hardware buffer og lægger de modtagne data over i uart\_ in \_ buffer. 
	\item[Interface] Tolker på brugerinputs og sender relevante kommandoer til Control- og read\_ pos-tasken.
	\item[read pos] Opdaterer target\_ pos med målets nuværende koordinater 60 gange i sekundet. I vores projekt læser den blot værdierne fra et array, men den kan sagtens udskiftes med en anden task der f.eks. tolker på input fra et kamera.
	\item[Control] Afvikler regulatorerne. Se mere indgående beskrivelse i sektion \ref{sec:control_task}. Denne task afvikles i realtid, med en frekvens på 600 Hz. 
	\item[log updater] Denne task opdaterer logfilen  med data fra "status" køen.
\end{description}


%Semaphorer på target_pos. 
%Ingen semaphorer på køer, undtagen uart out buffer


\subsubsection{Control task}
\label{sec:control_task}


\subsection{Test} 
% Test af SPI overførsel
% Test af realtidsafvikling af reguleringssløjferne, 
% lever vi op til kravene?
% 





\subsection{Delkonklusion}
