\section{Kravspecifikation}
\label{sec:kravspecifikation}
Kravene til systemet findes vha. reglerne for English Skeet, \citep{ES_regler},
samt det udleverede Pan \& Tilt-systems fysiske begrænsninger.

%%%%%%%%%%%%%%%%%%%%%P&T figur

Lerduens bane kan beregnes vha. de i reglerne givne værdier for affyringspunkt, forventet nedslagspunkt
og "Target Crossing Point" (punktet, som lerduen skal passere).
Med antagelsen om negligerbar luftmodstand kan en kasteparabel således udregnes ved interpolation af disse tre punkter.
Kasteparablen er givet ved ligning \ref{eq:ks:vektorparabel3d}, med orego som angivet på figur \ref{fig:ES}.
\begin{equation}
Pos\left( t \right) = 
\left( \begin{matrix} 
	x\left( t \right)  \\ 
	y\left( t \right)  \\ 
	z\left( t \right)  \end{matrix} \right) =
	 \left( \begin{matrix} 
	- 9,34\cdot t+5,5 \\
  32,851\cdot t-19,3 \\ 
 -{ 4,91\cdot t }^{ 2 }+5,473\cdot t+3,05\end{matrix} \right) [\text{m}]
\label{eq:ks:vektorparabel3d}
\end{equation}
En gennemgang af beregningerne der ligger bag kasteparablen findes i sektion \ref{sec:udregning_af_parabel}.

Med kasteparablen er det muligt at specificere systemets betingelser yderligere:
Lerduen bliver affyret med en hastighed på 34,589 \([\frac{m}{s}]\) i en vinkel på \(9,103^{\circ}\) ift. xz-planet jf. figur \ref{fig:ES}.
Set ovenfra bevæger målet sig som illustreret på figur \ref{fig:HH2D_para}, G er nedslagspunktet og er 52 [m] fra High-House 52 [m], D, samt PTS' placering, B.\\
\begin{figure}[h!]
\centering
\subfloat[Lerduens højde som funktion af afstanden til D.\label{fig:HH2D_para}]{%
	\begin{tikzpicture}[scale=0.8]
	\include*{./graphics/high_house_2D_parabola}
	\end{tikzpicture}
}
\subfloat[Lerduens bane projekteret på xy-planet.\label{fig:para_in_xy_plane}]{%
	\begin{tikzpicture}[scale=0.16]
	\include*{./graphics/parabola_in_xy_plane}
	\end{tikzpicture}
}
\caption[Lerduens parabel i 2D]{Viser parablen af lerduens bane i 2D.}
\end{figure}

Det anslås, at man kan nå at affyre de to skud mellem tidspunktet,
hvor lerduen når toppunktet og nedslagstidspunktet (punkt G), se \ref{eq:ks:nedslagstid}.
I dette tidsrum skal Pan \& Tilt-systemet sigte på lerduen så geværet kan ramme.
Der er altså et tidsmæssigt krav til reguleringen om en Settling Time for systemet,
der er lavere end tiden \(t_{s}\) angivet i \ref{eq:ks:settlingtime}.
\begin{align}
  t_{s} &= 0,557 [s] &\text{(Settling Time)}
  \label{eq:ks:settlingtime}
\end{align}
\begin{align}
  t_{n} &= 1,523 [s] &\text{(Nedslagstid)}
  \label{eq:ks:nedslagstid}
\end{align}

Haglene fra et 12 gauge haglgevær med Skeet Choke spredes så de dækker et område
med en diameter på 1,32 [m] på en afstand på 37 [m], \todo[inline,author=Mikael]{Kilde: wikipedia}. %http://en.wikipedia.org/wiki/Shotgun#Pattern_and_choke
Spredningsvinklen er dermed givet ved ligning \ref{eq:ks:spredning}.
\begin{align}
  Spredning &= \tan \left(\frac{1,32}{2} / 37 \right) \\
  &= 2,04 \degree
  \label{eq:ks:spredning}
\end{align}
Der er altså et krav om at geværet peger på lerduen med en præcision på \(\pm 1,02 \degree\).
Dette krav gælder for både pan-rammens vinkel (\(\phi\)) og tilt-rammens vinkel (\(\theta\)).
For en 1-radian reference giver dette altså en maksimal steady state tracking error (SSE) på 1,78 \%.

Tilt-rammen kan bevæge sig frit,
men pan-rammen kan pga. en stopklods kun rotere \(154 \degree\)\footnote{Målt eksperimentielt.}.
Dette giver altså en øvre og en nedre grænse for pan-rammens maksimale udsving,
som afhænger af reguleringen
- et for højt overshoot på pan-positionen kan give udsving, der ikke holder sig inden for grænserne.
