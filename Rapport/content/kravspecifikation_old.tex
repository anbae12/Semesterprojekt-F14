\section{Kravspecifikation}
\label{sec:kravspecifikation}
\todo[inline, author=Michael]{Tidligt udkast, vent venligst med at rette noget som helst ;)}
Kravene til systemet er givet ud fra reglerne for English Skeet. 
Beregningerne der ligger til grund for dette afsnit kan ses i appendix \textbf{5000}. 

PTS har monteret et gevær der skal kunne følge et mål i English Skeet. I reglerne er affyringspunktet, forventet nedslagspunktet og "Target crossing point" (som målet skal passere) givet. Målets kasteparabel er udregnet vha. interpolation af disse 3 punkter. Parablen er givet i \ref{eq:ks:vektorparabel3d}. 

\begin{equation}
Pos\left( t \right) = 
\left( \begin{matrix} 
	x\left( t \right)  \\ 
	y\left( t \right)  \\ 
	z\left( t \right)  \end{matrix} \right) =
	 \left( \begin{matrix} 
	- 9,34\cdot t+5,5 \\
  32,851\cdot t-19,3 \\ 
 -{ 4,91\cdot t }^{ 2 }+5,473\cdot t+3,05\end{matrix} \right) 
\label{eq:ks:vektorparabel3d}
\end{equation}


Normalt affyres geværet ca. når målet er ved toppunktet. 
PTS skal derfor sigte på målet senest når målet når toppunktet. Toppunktet er nået når den vertikale hastighed er nul.\footnote{Se appendix qp17}

\begin{align}
  t_{settling} &= 0,557 [s]
  \label{eq:ks:settlingtime}
\end{align}

Applikation er slut når målet rammer jorden. Nedslagstiden er givet ved:

\begin{align}
  t_{settling} &= 1,523 [s]
  \label{eq:ks:nedslagstid}
\end{align}

Haglene spredes så de dækker et område med diameter på 1,32 [m] på 37 meter. 
Dette gælder ved brug af Skeet-choke på geværet. Spredningsvinklen er givet ved:

\begin{align}
  Spredning &= \tan \left(\frac{1,32}{2} / 37 \right) \\
  &= 2.04 \degree
  \label{eq:ks:spredning}
\end{align}

PTS skal altså pege på målet med $\pm 1 \degree$ præcision. Dette gælder for både pan- ($\phi$) og tilt-rammen ($\theta$)

For tiltrammen er der ingen krav til overshoot, imens overshoot for pan er defineret udfra systemet fysiske begrænsninger. 




%\footnote{http://en.wikipedia.org/wiki/Shotgun#Pattern_and_choke}
