\section{Koordinattransformation}
\label{sec:koordinattransformation}

De kartesiske koordinater $P_k=[x, y, z]$ skal transformeres til sværiske koordinater $P_s=[\rho, \phi, \theta]$, hvor $\phi$ og $\theta$ bruges som vinkelbestemmelserne for hhv. pan og tilt.\\

Positionbestemmelsen af det kartesiske koordinatset som funktion af er udredt tidligere og fremgår i ligning \ref{eq:pf:vektorparabel3d1}. 

\begin{align}
\begin{split}
Pos\left( t \right) =\left( \begin{matrix} x \\ y \\ z \end{matrix} \right) =\left( \begin{matrix} 32,851\cdot t-19,3 \\ 9,34\cdot t-5,5 \\ -4,91{ \cdot t }^{ 2 }+5,473\cdot t+3,05 \end{matrix} \right) 
\label{eq:pf:vektorparabel3d1}
\end{split}
\end{align}
\todo[author=Anders,inline]{Mangler PTS offset husk målebånd.}

Koordinattransformationen fra kartesiske til sværiske kan gøres ud for nedenstående ligninger.


\begin{align}
\begin{split}
\rho &=\sqrt { { x }^{ 2 } +{ y }^{ 2 }+{ z }^{ 2 }} 
\\
\phi &={ tan }^{ -1 }\left( \frac { \sqrt { { x }^{ 2 }+{ y }^{ 2 } }  }{ z }  \right) 
\\
\theta &={ tan }^{ -1 }\left( \frac { y }{ x }  \right) 
\label{eq:sv_koordi}
\end{split}
\end{align}
hvor $\phi$ og $\theta$ er angivet i grader. \citep[Kap. 10.6, s. 598]{adam}.






\bigskip



Bestemmelse af PS for givet ligning..


PTS indsættes som reference. 

Henvis til matlab. 




Det er nødvendigt med en transformation mellem koordinatinputtet og 
