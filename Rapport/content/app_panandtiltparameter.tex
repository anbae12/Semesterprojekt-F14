\section{Beregning af inertimoment}
\label{sec:inertimomentberegning}
Dette appendix beskæftiger sig med beregningen af de teoretiske inertimomenter
for pan- og tilt-rammerne. Der henvises til figur \ref{fig:inerti_PTS}, og
beregningerne tager udgangspunkt i rammernes mål:
\({L_{1}} =0,292\) [m],
\({L_{2}} =0,280\) [m], \({L_{3}}= 0,42\) [m], \({L_{4}} =0,246\) [m], \({L_{pro}}=0,04\) [m].

\begin{figure}[!th]
\centering
\begin{tikzpicture}[scale=0.7]
\include*{./graphics/inerti_PTS}
\end{tikzpicture}
\caption[Skitse af pan \& tilt-rammerne]{Skitse af pan \& tilt-rammerne.}
\label{fig:inerti_PTS}
\end{figure}

Den teoritiske beregning er foretaget ud fra følgende antagelser og simplificeringer:
\begin{itemize}
\item Aluminimumsprofilen, 40x40L, har en massefylde på \(\rho=1,5\) [kg/m], \citep[Kap. 2 Side. 4]{alu_profil_desitet}.
\item Rammerne simplificeres som bestående af tynde stænger, således at hver ramme har to stænger, hvis bidrag
til inertimomentet kan beregnes som var de punktmasser forskudt i forhold til rotationsaksen \citep[Side. 254, ligning 10.36]{fund_of_physics},
vha. ligning \ref{eq:punktmasse_para}.
\begin{equation}
J={ J }_{ com }+M\cdot { h }^{ 2 }
\label{eq:punktmasse_para} 
\end{equation}
hvor \({J_{com}} = 0\) og \(h\) er afstand til punktmassen (stangen) fra rotationsaksen.
Inertimomentet for de stænger, der står vinkelret på rotationsakserne, kan beregnes vha. ligning \ref{eq:stang}
\citep[Side. 255, tabel 10-2e]{fund_of_physics}.
\begin{equation}
J=\frac { 1 }{ 12 } M\cdot { L }^{ 2 }
\label{eq:stang} 
\end{equation}
\end{itemize}

Med ovenstående formler er det muligt at finde pan og tilt-rammernes teoritiske inertimoment:
\begin{align}
\label{eq:inerti_tilt_pan}
\begin{split}
{ J }_{ tilt,1 } &= \left( \left( 1/12\cdot \rho \cdot { {L_{1}} }^{ 3 } \right) +\left( \rho \left( {L_{2}}-2\cdot {L_{pro}} \right) { \left( \frac { {L_{1}}-{L_{pro}}}{ 2 }  \right)  }^{ 2 } \right)  \right) \cdot 2
\\
 &= 0,0157499 \text{ [kg m$^2$]}
\\
{ J }_{ pan,1 }&=\left( 1/12\cdot \rho \cdot { { L }_{ 3 } }^{ 3 } \right) +\left( \rho \left( { L }_{ 4 }-{ L }_{ pro } \right) { \left( \frac { { L }_{ 3 }-{ L }_{ pro } }{ 2 }  \right)  }^{ 2 } \right) \cdot 2
\\
 &=0,0315708 \text{ [kg m$^2$]}
\end{split}
\end{align}

Tilt-rammens indflydelse på inertimomentet som pan-motoren skal rotere, er afhængig af tilt-rammens vinkel.
Når tilt-rammen er lodret er bidraget til pan-inertimomentet mindst, jf. ligning \ref{eq:punktmasse_para}, fordi afstanden
af to af stængerne til rotationsaksen (pan) er mindst. Ligeledes vil bidraget til pan-inertimomentet være størst når tilt-rammen er vandret.
Dette giver en kobling mellem pan og tilt, hvilket er uønsket under designet af reguleringssløjfen, som beskrevet i afsnit \ref{sec:matPTS}.
En simplificering er således at inkludere bidraget i modellen for pan som en konstant.
Man kan argumentere for, at denne konstant bør være gennemsnittet af bidragene for hele bevægelsen.
Denne ville kunne tilnærmes ud fra den ideelle responskurve for tilt-systemet, der kan beregnes ud fra den faste parabel, som gives som input.
Værdien ville således være baseret på ét specifikt parabelinput. En anden tilgang kunne være at vælge minimum-bidraget, og dermed
have en værdi der er nøjagtig for de tilt-vinkler, der er tæt på lodret, og mere og mere unøjagtig jo mere vandret tilit-vinklen er.
Sidstnævnte tilgang vælges, fordi tilt-vinklen i lerdueskydningen er tilnærmelsesvis lodret. Tilt-rammen roterer i vinkelintervallet \([4,03\degree \leq \phi \leq 12,3\degree]\) jf. lerduens toppunktstid samt nedslagstid på hhv. 0,557 [s] og 1,18 [s], indsat i ligning \ref{eq:sv_koordi}. Antagelsen om konstant bidrag til inertimoment er derfor nøjagtig.

Tilt-rammens bidrag til inertimomentet for pan bestemmes derfor ud fra, at to af tilt-rammens "stænger" er parallele med
rotationsaksen, og at de to stænger med længden \({L_{2}}\) er vinkelret på den, samt at de roterer om den.
Med ligningerne \ref{eq:punktmasse_para} og \ref{eq:stang}
kan tilt-rammens bidrag til pan-inertimomentet altså beregnes vha ligning \ref{eq:tiltOnPan}.
\begin{align}
\begin{split}
J_{pan,tilt_{min}}&=2\cdot{}\left(\frac{1}{12}\cdot{}\rho\cdot{}L_{2}^3
+\rho\cdot{}\left(L_1-2\cdot{}L_{pro}\right)\left(\frac{L_2-L_{pro}}{2}\right)^2\right)
\\
&=0,0146464 \text{ [kg m$^2$]}
\end{split}
\label{eq:tiltOnPan} 
\end{align}
Inertimomentet på pan-aksen er derfor givet ved summationen af pan-rammens inertimoment og tilt-rammens inertimoment,
som angivet i ligning \ref{eq:pan_inerti}.
\begin{align}
\begin{split}
{ J }_{ pan,1.1 } &= J_{ pan,1 }+J_ { pan,tilt_{ min} }
\\
&=0,0462172 \text{ [kg m$^2$]}
\end{split}
\label{eq:pan_inerti} 
\end{align}
Inertimomentet for tilt og pan er beregnet i ligning \ref{eq:inerti_tilt_pan}, mens det endelige inertimomentet for pan er beregnet i ligning \ref{eq:pan_inerti}.
Systemets gearing gør dog at inertimomenterne som motorerne belastes med, er mindre.
Gearingen i EMG30-motoren er i forholdet 30:1, mens gearingen til rammerne er i forholdet 1:3.
Ligning \ref{eq:gearing0} giver forholdet mellem det reflekterede inertimoment \(J_r\) (som motoren belastes med)
og inertimomentet \(J\) efter en gearing med forholdet \(N\) \citep{gear_inerti}.
\begin{equation}
J_r=\frac{J}{N^2}
\label{eq:gearing0}
\end{equation}
\(N\) kan altså bestemmes som multiplikationen af de to gearingsforhold, og er således lig med 10.
I ligning \ref{eq:inerti_tilt_pan_fak} er de reflekterede inertimomenter beregnet.
\begin{align}
\label{eq:inerti_tilt_pan_fak}
\begin{split}
{J_{tilt}}&=1,57499\cdot{10}^{-4} \text{ [kg m$^2$]}
\\
{J_{pan}}&=4,62172\cdot{10}^{-4} \text{ [kg m$^2$]}
\end{split}
\end{align}
