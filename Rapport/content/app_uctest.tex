\section{Test af mikrocontrollerens timing}
\label{sec:uctestappendix}
Dette appendiks beskriver test af mikrocontrollerens timing. 
Først testes afviklingstiden for SPI funktionerne, dernæst for Control-task'en. 


SPI funktionens afviklingstid blev testet ved at skifte en digital pin før og efter SPI-funktion blev kaldt.
Et oscilloskop af typen \textit{Agilent DSO-X 2024A} blev brugt til målingerne. 

SPI funktionens afviklingstid blev på oscilloskopet målt til 40,54 [\micro s].
I løbet af 10 parabelbevægelser blev observeret en afvigelse på \(\pm2\) [ns], hvilket svarer til måleusikkerheden. 
Altså overholder implementeringen kravet på afvikling indenfor 83,3 [\micro s], som opstillet i ligning \ref{eq:uc:spi-krav}.

Samme fremgangsmåde blev brugt til at undersøge timingen for Control-task'en,
som indeholder reguleringssløjferne. Oscilloskopet blev indstillet til \(500.000\) [Samples/s].
Kravet til afviklingen var: Reguleringsløjferne kører med 600 [Hz], samt at task'en afvikles inden næste periode. 
Dvs. en maks afviklingstid på \(\frac{1}{600}\) [s], da kravet på 600 [Hz] ellers ikke kunne opfyldes. 
Intervallet mellem hver afvikling af task'en blev målt for 60 afviklinger,
og resultaterne er vist i tabel \ref{tb:ctrltasktimingtest}.
\begin{table}[h!]
% Prøv multicolummn
\centering
\begin{tabu}{l|[1.25pt]r|r|r|r}
 & Max  & Min & Median & Gennemsnit  \\ \tabucline[1.25pt]{-}
Målt interval & 1,672 [ms] & 1,662 [ms] & 1,667 [ms] & 1,667 [ms] \\ 
\hline 
Afvigelse & 5,33 [\micro s] & 4,66 [\micro s] & 0,66 [\micro s] & 0,13 [\micro s] \\
 & 0,41\% & 0,54\% & 0,03\%  & 0,03\% \\
\end{tabu} 
\caption[Interval mellem afvikling af Control task]{Interval mellem hver afvikling af Control-task. Måleusikkerheden er $\pm4$ [\micro s].}
\label{tb:ctrltasktimingtest}
\end{table}
%\todo[inline,author=Nikolaj, color=green]{Din afvigelse siger at du mangler et decimal...}

Afvigelserne for medianen og gennemsnittet ligger indenfor måleusikkerheden på $\pm4$  [\micro s].
For max. og min. værdierne er afvigelsen på max. 133\% af måleusikkerheden,
og det vurderes at denne kan negligeres. 

Samme datasæt blev brugt til at beregne afviklingstiden for task'en, koordinattransformationen og selve reguleringssløjfen. 
Disse data er vist i tabel \ref{tb:ctrl task runtime test}. %Bemærk usikkerheden på $4  [\micro s]$.

%500KSa/s
\begin{table}[h!]
\centering
\begin{tabu}{l|[1.25pt]r|r|r|r}
Afviklingstid [\micro s] & Max  &  Min & Median & Gennemsnit  \\ \tabucline[1.25pt]{-}
Ctrl task & 360  & 326  & 332  & 338  \\ \hline 
Koordinat transformation & 228  & 202  & 206  & 212  \\
\hline 
PID-regulering & 40  & 30  & 32  & 34 \\
\end{tabu} 
\captionsetup{width=0.8\textwidth}
\caption[Afviklingstiden for Control task]{Afviklingstiden for Control task, Koordinattransformation og PID-regulering. Måleusikkerheden er $\pm4$ [\micro s].}
\label{tb:ctrl task runtime test}
\end{table}

Dataene viser at Control task'en blev afviklet før sin deadline - med en margin på min. 1,294 [ms].
%\todo[inline,author=Nikolaj, color=green]{Skal man ikke tænke mindste målte interval - måleusikkerheden - længste afviklingstid? - Indregn worst case}
Variationen i afviklingstiden skyldes koordinattransformationen og PID-reguleringen. Den maksimale variation af afviklingstiden udgør $ \frac{360 - 326}{1667} = 2,04\% $ af periodetiden, hvilket kan tolereres. 
