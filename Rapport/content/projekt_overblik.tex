\section{Projektoverblik}
\label{sec:projektoverblik}
Som det fremgår af projektoplægget skal projektet opdeles som vist på 
figur \ref{fig:overview_openloop_PTS}. 
Det ligger altså udenfor dette projekt at bestemme den overordnede opdeling.
Projektoplægget stiller ydermere følgende krav til implementeringen:
\begin{itemize}
\itemsep1pt
  \item Regulatorerne skal implementeres på én mikroprocessor.
  \item Der skal benyttes SPI kommunikationen imellem mikroprocessoren og FPGA’en.
  \item FPGA’en skal styre PWM-signalerne til motorerne.
  \item FPGA’en skal benyttes til bestemmelse af motorernes position vha. encoderne.
\end{itemize}

\bigskip

\begin{figure}[!th]
\centering
\begin{tikzpicture}[auto, node distance=1cm,>=latex']
\include*{./graphics/PTSoverview}
\end{tikzpicture}
\caption[Principskitse af PTS]{Principskitse af PTS med farvekoder.}
\label{fig:overview_openloop_PTS}
\end{figure}
De grønne kasser på figur \ref{fig:overview_openloop_PTS}
repræsenterer mikrocontrolleren som vha. UART kommunikerer med brugeren. \\
FPGA'en er repræsenteret vha. en lilla kasse som er bindeledet mellem mikrocontrolleren og de to DC motorer. 
FPGA'en og mikrocontrolleren kommunikerer sammen vha. SPI.
FPGA'en genererer to PWM-signaler, som driver hhv. H-broen og motor for
pan og H-broen og motor for tilt. DC-motorene sender deres encoder feedback til FPGA'en.