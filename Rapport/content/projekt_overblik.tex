\section{Projektoverblik}
\label{sec:projektoverblik}
%Anders afsnit:
%\todo[inline,author=Anders]{Der må gerne gives feedback.. jeg synes selv jeg er lidt på bar bund med hvad der skal stå}
%Inden rapportens kommende afsnit introdukseres et kort overblik af projektets komponenter og hvordan de kommunikerer med hinanden.\\
%Som der ses på nedenstående figur \ref{fig:overview_openloop_PTS}, er de forskellige komponenter opdelt i forskellige farver.\\
%De grønne kasser repræsenterer mikrocontrolleren som vha. UART kommunikerer med brugeren. \\
%FPGA'en er repræsenteret vha. en lilla kasse som er bindeledet mellem mikrocontrolleren og de to DC motorer. 
%FPGA'en og mikrocontrolleren kommunikere sammen vha. SPI. FPGA'en genererer to PWM-signaler, som driver hhv. H-broen og motor for pan og H-broen og motor for tilt. DC motorene sender deres encoder feedback til FPGA'en.

%Mikkels:

Af projektoplægget er givet, at projektet skal opdeles som vist på 
figur \ref{fig:overview_openloop_PTS}. 
Det ligger altså udenfor dette projekt at bestemme den overordnede opdeling.
Projektoplægget stiller ydermere følgende krav til implementeringen:
\begin{itemize}
  \item Regulatorerne skal implementeres på én microprocessor.
  \item Der skal benyttes SPI kommunikationen imellem microprocessoren og FPGA’en.
  \item FPGA’en skal styre PWM signalerne til motorerne.
  \item FPGA’en skal benyttes til at bestemmes motorernes position via encoderne.
\end{itemize}

\bigskip

\begin{figure}[!th]
\centering
\begin{tikzpicture}[auto, node distance=1cm,>=latex']
\include*{./graphics/PTSoverview}
\end{tikzpicture}
\caption[Principskitse af PTS]{Principskitse af PTS med farvekoder.}
\label{fig:overview_openloop_PTS}
\end{figure}

\todo[inline, author=Michael]{Vi skal beslutte hvad vi vil med dette afsnit: Læseguide? Blokopdeling? Indledning? Krav til enkelte dele? Overblik over valgt løsning? \\ Det afhænger vel af hvilken rapportstruktur vi vælger.}