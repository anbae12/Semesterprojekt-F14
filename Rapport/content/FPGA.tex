\section{FPGA}
\label{sec:FPGA}

\todo[inline]{Mikkel: Er slet ikke færdigt, så lad venligst være med at rette eller andet.}
FPGA'en er leddet mellem microcontrolleren og motorene på PTS. 
Kommunikationen mellem FPGA og microcontroller skal ske ved hjælp af SPI. 
FPGA'en skal producere PWM signal til motorerne på PTS og skal kunne tælle motorenes rotationer. 
FPGA'en er i projeket et BASYS board \footnote{DIGILENT BASYS 2}, der programmeres ved brug af VHDL.

\subsection{Struktur}
Det er forsøgt at bygge funktionaliteter op ved at skrive generiske komponenter. 

\subsection{PWM}
Der skal fra FPGA'en genereres PWM signaler til at styre motorerne på PTS.
\subsubsection*{Valg af PWM frekvens}
Det ønskedes at have en PWM frekvens, der ikke ligger i det hørbare område. 
Det ønskedes også at have en god opløsning på PWM. 
Ved 11 bit og brug af BASYS boardets interne clock på 50 MHz bliver PWM frekvensen passende høj:
\begin{equation}
  \frac{2^{11}}{50 MHz} = 24.41 KHz 
\end{equation}

\begin{figure}[!th]
\centering
\begin{tikzpicture}[node distance = 5 cm,scale=1]
\include*{./graphics/FPGA_blok}
\end{tikzpicture}
\caption[tekst i indholdsfortegnelsen]{figurtekst.}
\label{fig:FPGA_blok}
\end{figure}
\todo[author=Anders,inline]{Jeg skal lige have tjekket op med nikolaj omkring forbindelsen til Decoder til output decoder.}



%%% FORMAT SPI

Fra Mikrokontroller til FPGA format:
\begin{figure}[th!]
\centering
\begin{tabular}{c|c|c|c|c}
1 motorbit &1 retningsbit & 1 sæt PWMbit & 2 NOPbit & 11 PWM dutycyclebit

\end{tabular}
\captionsetup{type=figure}
\caption[tekst i indholdsfortegnelsen]{tabeltekst}
\label{tb:}
\end{figure}




Fra FPGA til Mikrokontroller format:
\begin{figure}[th!]
\centering
\begin{tabular}{c|c|c|c}
1 motorbit & 4 NOPbit & 11 decoderbit

\end{tabular}
\captionsetup{type=figure}
\caption[tekst i indholdsfortegnelsen]{tabeltekst}
\label{tb:}
\end{figure}