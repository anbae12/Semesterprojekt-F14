\section{Reguleringsdesign}
\label{sec:kontrollerdeign}

Måske skal den første del placeret et andet sted. \\

Litteratur/DCmotor/5.4



DC motor puls fundet konstanter.
\subsection{2. ordens differentialligning af DC motorene}
Den anden ordens differentialligning, som beskriver DC motoren med følgende simplificeringer er givet i ligning \ref{eq:generel_andenordens}.
\begin{itemize}
\item bla 
\item bla 
\end{itemize}

\begin{align}
\begin{split}
v\left( t \right) =\frac { { L }_{ m }\cdot { J }_{ m+PTS } }{ { k }_{ b } } \frac { { d }^{ 2 }\omega  }{ dt } +\frac { { R }_{ m }\cdot { J }_{ m+PTS } }{ { k }_{ b } } \frac { d\omega  }{ dt } +\omega \cdot B\label{eq:generel_andenordens}
\end{split}
\end{align}

Laplace med start betingelserne t=0 w'(t)=0 og w''(t)=0,
\begin{align}
\begin{split}
\mathcal{L}\left( v\left( t \right)  \right) &=\mathcal{L}\left( \frac { { L }_{ m }\cdot \left( { J }_{ m }+{ J }_{ PTS } \right)  }{ { k }_{ b } } \frac { { d }^{ 2 }\omega  }{ { dt }^{ 2 } } +\frac { { R }_{ m }\cdot \left( { J }_{ m }+{ J }_{ PTS } \right)  }{ { k }_{ b } } \frac { d\omega  }{ dt } +\omega \cdot B \right) 
\\
v\left( s \right) &=\frac { { L }_{ m }\cdot \left( { J }_{ m }+{ J }_{ PTS } \right)  }{ { k }_{ b } } { s }^{ 2 }\cdot \omega \left( s \right) +\frac { { R }_{ m }\cdot \left( { J }_{ m }+{ J }_{ PTS } \right)  }{ { k }_{ b } } { s }\cdot \omega \left( s \right) +B\cdot \omega \left( s \right) 
\label{eq:generel_andenordens_laplace}
\end{split}
\end{align}



Ligning \ref{eq:generel_andenordens_laplace} faktoriseret, hvorefter det er muligt at finde transferfunktionen for DC motoren. 
\begin{align}
\begin{split}
G\left( s \right) =\frac { \omega \left( s \right)  }{ v\left( s \right)  } =\frac { 1 }{ \frac { { L }_{ m }\cdot \left( { J }_{ m }+{ J }_{ PTS } \right)  }{ { k }_{ b } } { s }^{ 2 }+\frac { { R }_{ m }\cdot \left( { J }_{ m }+{ J }_{ PTS } \right)  }{ { k }_{ b } } s+B } 
\label{eq:generel_tf}
\end{split}
\end{align}

hvor 
\(K_b = 0.517\), \(B=0.00319\), \(R_m=5.21\), \(J_m=8.26*10^{-4}\), \(Tf=5.71*10^{-2}\), \(Lm=2.667*10^{-3}\), \(J_{tilt}=1.57499*10^{-4}\) og \(J_{pan}=3.15708*10^{-4}\).\\
\todo[inline,author=Anders]{Der mangler enheder på tallene ovenfor.}
Inertimomenterne for pan og tilt, som der fremgår fra tidligere, har forskellig påvirkning på DC motor modellen. De indelige transferfunktioner for DC motorene til hhv. pan og tilt-rammerne er givet i ligning \ref{eq:pan_tilt_tf_result}. 
\begin{align}
\begin{split}
G_{ pan }\left( s \right) &=\frac { \omega _{ pan }\left( s \right)  }{ v_{ pan }\left( s \right)  } =\frac { 1 }{ \frac { { L }_{ m }\cdot  \left( { J }_{ m }+{ J }_{ pan } \right) }{ { k }_{ b } } { s }^{ 2 }+\frac { { R }_{ m }\cdot  \left( { J }_{ m }+{ J }_{ pan } \right) }{ { k }_{ b } } s+B } 
\\
&=\frac { 1 }{ 5,89\cdot { 10 }^{ -6 }\cdot { s }^{ 2 }+0,01151\cdot s+0,00319 } 
\\
G_{ tilt }\left( s \right) &=\frac { \omega _{ tilt }\left( s \right)  }{ v_{ tilt }\left( s \right)  } =\frac { 1 }{ \frac { { L }_{ m }\cdot  \left( { J }_{ m }+{ J }_{ tilt } \right) }{ { k }_{ b } } { s }^{ 2 }+\frac { { R }_{ m }\cdot \left( { J }_{ m }+{ J }_{ tilt } \right) }{ { k }_{ b } } s+B } 
\\
&= \frac { 1 }{ 5,073\cdot { 10 }^{ -6 }\cdot { s }^{ 2 }+0,009911\cdot s+0,00319 } 
\label{eq:pan_tilt_tf_result}
\end{split}
\end{align}

\subsection{Krav til systemet}
Generelt krav
\begin{itemize}
\item P.O. = [\%]
\item Settlingtime = [s]
\item ESS = 
\end{itemize}


\subsection{PI, PD eller PID}
analyse af rlocus()\\
eventuel lille diskussion, hvorfor vi vælger den kontroller som vi gør. 
???