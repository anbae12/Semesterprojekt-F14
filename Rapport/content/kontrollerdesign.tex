\section{Regulator}
\label{sec:kontrollerdeign}
Det er valgt at designe regulatoren efter processen illustreret i figur \todo{fig}.%\ref{fig:designprocess}.
\missingfigure{fig:designprocess}
Formålet med reguleringen er som beskrevet i afsnit \ref{sec:problemformulering},
at kontrollere pan- og tilt-rammernes position, så de tracker en lerdue.
Dette gøres ved styring af deres hastighed, ved at justere spændingsfaldene over de
to DC-motorer. Spændingsfaldene styres af PWM-generatorer, og regulatoren
kan derfor styre motorernes hastighed ved at vælge PWM-signalernes duty cycles.

I afsnit \ref{sec:kravspecifikation} opstilles kravene til systemets respons.
Disse er opsummeret nedenfor.
\begin{itemize}
\item \(t_{s} <= 0,557 \mathrm{\left[s\right]}\) (Settling Time)
\item \(SSE <= 1 \degree\) (Steady State Error i grader)
\item \(P.O. (pan) <= 5 \%\) (Overshoot i procent)
\end{itemize}


\subsection{Design af PI, PD eller PID}
\begin{figure}[!th]
\centering
\begin{tikzpicture}[auto, node distance=2.6cm,>=latex']
\include*{./graphics/digitalkontroller1}
\end{tikzpicture}
\caption[tekst i indholdsfortegnelsen]{figurtekst}
\label{fig:}
\end{figure}

\begin{figure}[!th]
\centering
\begin{tikzpicture}[auto, node distance=2.6cm,>=latex']
\include*{./graphics/digitalkontroller2}
\end{tikzpicture}
\caption[tekst i indholdsfortegnelsen]{figurtekst}
\label{fig:}
\end{figure}

\begin{figure}[!th]
\centering
\begin{tikzpicture}[auto, node distance=2.6cm,>=latex']
\include*{./graphics/digitalkontroller3}
\end{tikzpicture}
\caption[tekst i indholdsfortegnelsen]{figurtekst}
\label{fig:}
\end{figure}