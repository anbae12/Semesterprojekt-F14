\section{Regulator}
\label{sec:kontrollerdeign}
Det er valgt at designe regulatoren efter processen illustreret i figur \ref{fig:designproces}.
\begin{figure}[!th]
\centering
\include*{./graphics/designproces}
\caption[Designprocessen]{Designprocessen, \citep{reg_modern_control_systems}.}
\label{fig:designproces}
\end{figure}
Formålet med reguleringen er som beskrevet i afsnit \ref{sec:problemformulering},
at kontrollere pan- og tilt-rammernes position, så de tracker en lerdue.
Dette gøres ved styring af deres hastighed, ved at justere spændingsfaldene over de
to DC-motorer. Spændingsfaldene styres af PWM-generatorer, og regulatoren
kan derfor styre motorernes hastighed ved at vælge PWM-signalernes duty cycles.

I afsnit \ref{sec:kravspecifikation} opstilles kravene til systemets respons.
Disse er for en 1-radian reference opsummeret nedenfor.
\begin{itemize}
\item \(t_{s} \leq 0,557 \mathrm{\left[s\right]}\) (Settling Time)
\item \(SSE \leq 1,78 \%\) (Steady State Tracking Error)
\item \(P.O. \leq 134 \%\) (Overshoot i procent)
\end{itemize}

Det er fastlagt i projektoplægget, at regulatoren skal implementeres på mikrocontrolleren.
Da systemet som input modtager kartesiske koordinater,
skal der også på mikrocontrolleren foregå en koordinattransformation
til den logiske vinkelrepræsentation med sfæriske koordinater.
Systemets konfiguration består altså af en koordinattransformation,
en regulering, en aktuering og en positionsmåling, som illustreret
i figur \ref{fig:digitalkontroller1}.
Bemærk at denne konfiguration er for ét SISO-undersystem (enten pan eller tilt),
og at mikrocontrolleren skal regulere begge SISO-undersystemer.
\begin{figure}[!th]
\centering
\begin{tikzpicture}[auto, node distance=2.6cm,>=latex']
\include*{./graphics/digitalkontroller1}
\end{tikzpicture}
\caption[Systemkonfiguration]{Systemkonfiguration}
\label{fig:digitalkontroller1}
\end{figure}
Som beskrevet i afsnit \ref{sec:problemformulering},
så er input-samplingen fastlagt til at foregå med en frekvens på 60 [Hz].
Men som illustreret på figur \ref{fig:digitalkontroller1}, så skal reguleringssløjfen
"køres" (sample) med en frekvens \(f_s=\frac{1}{T_s}\), der ikke nødvendigvis er 60 [Hz].
Dvs. A/D- og D/A-konverteringerne skal foregå med frekvensen \(f_s\).

Der er overordnet to strategier til valg af samplingfrekvensen \(f_s\) hvormed
reguleringssløjfen skal køre.
Hvis man designer en kontinuert regulator til det kontinuerte domæne, så
skal diskretiseringen af controlleren være så tæt på den kontinuerte regulator som muligt.
Det vil sige, samplingfrekvensen skal vælges så høj som mulig.
Hvis man derimod designer en diskret regulator til det diskrete domæne,
så er diskretiseringen allerede foretaget inden designet af regulatoren.
Dvs. man finder en diskretiseret model af det fysiske system inden designet af regulatoren.
Kravet til diskretiseringen af åbensløjfeoverføringsfunktionerne er, at den diskrete repræsentation
skal være tilfredsstillende tæt på de kontinuerte overføringsfunktioner.
Hvis den diskrete overføringsfunktion eksempelvis afviger 20 \% fra den kontinuerte, ville man
sandsynligvis overveje at benytte en højere samplingfrekvens til diskretiseringen.
Sammenligningen af den diskrete overførselsfunktion og den kontinuerte overføringsfunktion
kan være både i tidsdomænet (fx steprespons) og i frekvensdomænet (frekvensrespons, fx. Bode-plots).




\begin{figure}[!th]
\centering
\begin{tikzpicture}[auto, node distance=2.6cm,>=latex']
\include*{./graphics/digitalkontroller2}
\end{tikzpicture}
\caption[tekst i indholdsfortegnelsen]{figurtekst}
\label{fig:}
\end{figure}

\begin{figure}[!th]
\centering
\begin{tikzpicture}[auto, node distance=2.6cm,>=latex']
\include*{./graphics/digitalkontroller3}
\end{tikzpicture}
\caption[tekst i indholdsfortegnelsen]{figurtekst}
\label{fig:}
\end{figure}