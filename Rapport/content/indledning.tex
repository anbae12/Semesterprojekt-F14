\section*{Indledning}
Trackingsystemer spiller en stadig større rolle i vores hverdag og ses ofte i gadebilledet i form 
af f.eks. overvågningskameraer.
Fælles for mange af sådanne trackingsystemer er, at de bruger et Pan \& Tilt System til at følge deres mål. 

I dette projekt udvikles et Pan \& Tilt trackingsystem, som følger en lerdues bevægelse i 
luften. Systemet dimensioneres til brug i en English Skeet konkurrence. Kravene til systemet analyseret i første del af rapporten. 

For at kunne designe en passende regulator til systemet skal udledes en matematisk model. Dette gøres i anden del af rapporten. 

I tredje del af rapporten diskuteres implementeringen af regulatorerne og diverse hjælpe-funktioner på mikrocontrolleren. Der beskrives også hvordan en FPGA bruges til at generere PWM-signaler og læse motorernes position. 

Inden rapporten konkluderes, beskrives justeringen/tuningen af regulatorerne. Dette er fjerde del af rapporten, som også indeholder tests af systemets endelige performance. 
%For at udvikle et sådant system skal passende regulatore designes og implementeres på en mikrocontroller. 
%Desuden skal kommunikation mellem pan \& tilt systemet og mikrocontrolleren oprettes.
%En FPGA  bruges til at læse systemets position og generere ønsket PWM-signal til motorerne. 
%Denne skal kunne kommunikere med microcontrolleren vha. en FPGA. 
Projektet udføres i overensstemmelse med det tilhørende projektoplæg, der er vedlagt 
som bilag.
% \todo[inline,color = pink, author= Mikkel]{1. forsøg på indledning. \\Michael: Har ændret lidt hist og her - mangler stadig lidt arbejde}
