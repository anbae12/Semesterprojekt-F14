\section{Serial Peripheral Interface}
Kommunikationen mellem mikrocontolleren og FPGA’en foregår med SPI (Serial Peripheral Interface). SPI understøtter full-duplex kommunikation mellem en master og en eller flere slaver
%Det er en standard hvor der bliver overført data mellem en master og en eller flere slaver. 
%Overførelsen sker med full duplex, så der bliver sendt data både fra masteren og slaven på samme tid. 
Opstillingen med en master og en slave kan ses på figur \ref{fig:SPImasterslave}.
\begin{figure}[!th]
\centering
\begin{tikzpicture}[scale=0.8]
\include*{./graphics/MISOMOSI}
\end{tikzpicture}
\caption[SPI protokol]{Skitse af SPI-kommunikation mellem mikrokontroller og FPGA.}
\label{fig:SPImasterslave}
\end{figure}

Der er fire ledninger mellem masteren og slaven: Slave Select (SS), Serial Clock (SCLK), Master Output Slave Input (MOSI) og Master Input Slave Output (MISO). SS bruges til at vælge hvilken slave der skal overføres data til/fra. Den kan også sammenlignes med en chip select. SCLK generes af masteren; SS, MOSI og MISO kører synkront med dette signal.