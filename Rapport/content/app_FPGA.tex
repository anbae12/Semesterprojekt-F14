\section{FPGA'ens komponenter}
\label{sec:fpgaappendix}
Dette appendiks forklarer alle komponenter på FPGA'en.
Et forsimplet diagram over komponenterne på FPGA'en ses på figur \ref{fig:FPGA_blok}.


\textbf{Tick counter}\\
\textbf{Input:}
Signaler fra encoder på PTS. Reset signal fra tick reset.

\textbf{Output:} Positionen på hver af motorerne angivet i antal ticks mellem 0 
og 1079. Retningen, som motorerne roterer i. 

Da encoderne på PTS er quadrature encodere er det muligt at bestemme både retning og 
tælle rotationer på motoren.
Tæller positionen op eller ned afhængigt af inputtet fra encoderne. 
Tæller ned fra 0 til 1079.
For at kalibrere positionen er det nødvendigt for tick counter komponenten at få 
et reset signal fra tick reset komponenten. 
Når dette reset signal modtages nulstilles positionen og er herved kalibreret.

\textbf{Tick reset}\\
\textbf{Input:} Signaler fra Hall sensorer på PTS. Rotationsretning.\\
\textbf{Output:} Reset signal.

\begin{figure}[!th]
\centering
\include*{./graphics/hall_sensor_signal}
\caption[Signal fra Hall sensor]{Signalet fra Hall sensor, når magneten på en af rammerne køres hen over. Nulstillingen sker første gang reset signalet udsendes.}
\label{fig:hall_sensor_signal}
\end{figure}

Som beskrevet tidligere er der behov for et reset signal til positionen. 
Dette skal gøres ved hjælp af Hall sensorer på PTS. 
Det viser sig dog, at Hall sensorernes output ikke kun er højt på ét encodertick. 
Derfor kan signalet fra Hall sensorerne ikke i sig selv bruges som reset signal.
PTS rotationsretningen tages derfor med i betragtning.
Hvis PTS roterer i én retning sættes reset signalet højt på opadgående 
flanke på Hall sensor outputtet, mens reset signalet sættes højt på nedadgående 
flanke i den anden retning. Reset signalet udsendes derved på samme sted, som vist på figur \ref{fig:hall_sensor_signal}.
Hermed fås det smallest mulige område, hvor reset signalet er højt.
Præcisionen på denne kalibrering er \(\pm1\) tick.

\textbf{Display}\\
\textbf{Input:} Motorposition fra tick counter. Skaleret clock signal fra clock 
scaler.

\textbf{Output:} Signal der styrer displayet.

Modtager motorpositionen fra tick counter komponenten og viser det på de fire 8-digit-displays 
på BASYS boardet. Bruger det skalerede clock signal til at multiplexe mellem de 
fire displays.

\textbf{Clock scaler}\\
\textbf{Input:} BASYS boardets interne clock.\\
\textbf{Output:} Skaleret clock signal

Skalerer clock signalet.

\textbf{Input decoder}\\
\textbf{Input:} 16 bit data fra SPI.\\
\textbf{Output:} PWM-kanal og tilhørende PWM-duty cycle (11 bit).

Hvis setPWM bit er højt skal PWM-kanal og tilhørende PWM-duty cycle opdateres.
Opdateringen sker ved at tage 2 MSB og lave dem til et PWM-kanal output. De 11 
LSB sendes som PWM-duty cycle output.

\textbf{PWM}\\
\textbf{Input:} PWM-kanal og tilhørende PWM-duty cycle (11 bit).\\
\textbf{Output:} PWM til PTS.

Sørger for at generere PWM-signaler til PTS ud fra PWM-kanaler og PWM-duty cycles.

%\subsection*{Output decoder}
\textbf{Output decoder}\\
\textbf{Input:} Motorposition fra tick counter.\\
\textbf{Output:} 16 bit data.

Sørger for skiftevis at sætte motorposition for de to motorer som LSB.
MSB bruges til at angive, hvilken motor positionen er til.

\textbf{SPI}\\
\textbf{Input:} 16 bit data fra output decoder. \\%16 bit data udefra.\\
\textbf{Output:} 16 bit SPI data ud.

Agerer som SPI slave i kommunikation med mikrocontrolleren.
Modtager SPI data udefra. Sørger samtidigt for at sende de 16 bit data fra output decoder.


