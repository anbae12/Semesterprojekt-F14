
\setcounter{page}{1}
\section*{Abstract}
%This paper describes the development of a Pan \& Tilt System for the tracking of a moving
This paper describes the development of a Pan \& Tilt System for tracking a moving
target. The system is designed for tracking the parabolic movement of
a clay target in an English Skeet competition.
Obtaining the position of the clay target is not a part of this project. 

A mathematical model of the physical dynamics of the Pan \& Tilt is
%developed and used as basis for the controller design.
derived and used as basis for the controller design.

Two digital PID-controllers are designed: one for the pan system and one for the tilt system.
The controllers are designed in the discrete domain, and therefore a discretized model
of the continuous part of the system is first obtained.
The controllers utilize position feedback from 
%the encoders mounted on 
the two DC motors
rotating the pan and tilt frames.
The controller design process is aided by simulations of the Pan \& Tilt systems respectively,
so as to speed up the design process and take into account dead zone nonlinearities.

The control algorithms are implemented on a microcontroller, while
the feedback of the DC motors is decoded and translated to positions on an FPGA.
The FPGA thus provides the position feedback for the microcontroller.
The PWM signals for the DC motors are generated by the FPGA, while the PWM duty cycles are calculated by the control algorithms.
The microcontroller communicates with the FPGA through SPI.

The real-time requirements of the digital controllers are met through the use
of a preemptive priority scheduling algorithm provided by the FreeRTOS operating
system.
% running on the microcontroller.

Test results are presented,
verifying the mathematical model of the dynamics of the Pan \& Tilt System
and the real-time performance of the microcontroller.
%Furthermore, the tracking performance of the Control System to an input signal
%of the English Skeet competition is measured.
Furthermore, the performance of the system tracking a clay target in a English Skeet competition is measured.

The usefulness of the implemented digital controllers to the performance requirements
of the application is discussed. Finally, options for further development of the system is discussed. 

It is concluded that the developed Pan \& Tilt System
is not always able to accurately track the parabolic movement of the clay target while
meeting the time requirements of the competition.
However, in all the experiments with the optimized system,the measurements show
that the system is able to accurately track the
clay target for 0.42 seconds after the peak of it's parabolic movement.

%%%
%%%
%%%This abstract is not finished at all.
%%%
%%%This paper describes the development of a pan \& tilt system tracking a clay 
%%%pigeon moving in the air. 
%%%The tracking of the clay pigeon to cartesian coordinates is not included in this 
%%%project. This paper concentrates on making the pan \& tilt system track inputs 
%%%given to the system with a rate of 120 [Hz] of a parabola.
%%%The project (and this paper) is divided into subparts: 
%%% 
%%%\begin{itemize}
%%%  \item \textbf{Application}
%%%  % - Defining the application, the requirements for 
%%%  %the system and the goals of the project.
%%%  \item \textbf{System identification} 
%%%  %- Identification of the system and it's 
%%%  %corresponding transferfunction. Analysis of the controller needed and the 
%%%  %requirements needed for good performance. Coordinate transformation from
%%%  %cartesian to spherical coordinates.
%%%  \item \textbf{Implementation}
%%%  % - Description of the implementation on the 
%%%  %microcontroller and the FPGA.
%%%  \item \textbf{Tuning of the controller} 
%%%  %- Tuning of the controller to get the 
%%%  %wanted performance.
%%%  \item \textbf{Tests} 
%%%  %- Test of the performance of the pan \& tilt system with 
%%%  %the controller.
%%%  \item \textbf{Discussion} 
%%%  %- Discussion of the project and the performance of 
%%%  % the system.
%%%\end{itemize}
%%%
%%%The pan \& tilt system will be uptimized for shooting in English Skeet. For a 1 
%%%radian reference the settling time should be less than 0.557 [s] and steady 
%%%state error less than 1,78\%.
%%%
%%%The systems transferfunctions were obtained by describing the system with a 
%%%mathematical model based on tests of a DC motor. The model was...... It was 
%%%found that a PID controller was needed....
%%%
%%%Implementation of the needed functionality was made on a microcontroller and a 
%%%FGPA. The microcontroller in charge of the both the controlling and 
%%%communication with the FPGA using SPI. 
%%%On the microcontroller FreeRTOS, an operating system using preemptive scheduling, was used 
%%%to ensure hard realtime perfomance.
%%%The FPGA was programmed to keep count of the motorrotations, to produce PWM 
%%%signals to the motors and to communicate with the microcontroller as a SPI slave.
%%%
%%%\todo[inline,color=pink,author=Mikkel]{Bør skrives færdig når konklusion osv. er skrevet.}