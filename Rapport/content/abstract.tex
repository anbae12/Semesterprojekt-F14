
\setcounter{page}{1}
\section*{Abstract}

This paper describes the development of a pan \& tilt system tracking a clay 
pigeon moving in the air. 
The tracking of the clay pigeon to cartesian coordinates is not included in this 
project. This paper concentrates on making the pan \& tilt system track inputs 
given to the system with a rate of 120 [Hz] of a parabola.
The project (and this paper) is divided into subparts: 
 
\begin{itemize}
  \item \textbf{Application}
  % - Defining the application, the requirements for 
  %the system and the goals of the project.
  \item \textbf{System identification} 
  %- Identification of the system and it's 
  %corresponding transferfunction. Analysis of the controller needed and the 
  %requirements needed for good performance. Coordinate transformation from
  %cartesian to spherical coordinates.
  \item \textbf{Implementation}
  % - Description of the implementation on the 
  %microcontroller and the FPGA.
  \item \textbf{Tuning of the controller} 
  %- Tuning of the controller to get the 
  %wanted performance.
  \item \textbf{Tests} 
  %- Test of the performance of the pan \& tilt system with 
  %the controller.
  \item \textbf{Discussion} 
  %- Discussion of the project and the performance of 
  % the system.
\end{itemize}

The pan \& tilt system will be uptimized for shooting in English Skeet. For a 1 
radian reference the settling time should be less than 0.557 [s] and steady 
state error less than 1,78\%.

The systems transferfunctions were obtained by describing the system with a 
mathematical model based on tests of a DC motor. The model was...... It was 
found that a PID controller was needed....

Implementation of the needed functionality was made on a microcontroller and a 
FGPA. The microcontroller in charge of the both the controlling and 
communication with the FPGA using SPI. 
On the microcontroller FreeRTOS, an operating system using preemptive scheduling, was used 
to ensure hard realtime perfomance.
The FPGA was programmed to keep count of the motorrotations, to produce PWM 
signals to the motors and to communicate with the microcontroler as a SPI slave.

\todo[inline,color=pink,author=Mikkel]{Bør skrives færdig når konklusion osv. er skrevet.}