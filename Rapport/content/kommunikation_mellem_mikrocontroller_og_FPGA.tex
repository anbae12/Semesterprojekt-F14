\section{Kommunikation mellem mikrocontroller og FPGA}

Kommunikationen mellem Microcontolleren og FPGA’en sker med SPI. SPI står for Serial Peripheral Interface. Det er en protokol hvor, der bliver overført data mellem en master og en eller flere slaver. Overførelsen sker i full duplex mode, så der bliver sendt data både fra masteren og slaven på samme tid.
Der er fire forbindelse mellem masteren og slaverne: Slave Select, Serial CLock (SCLK), MOSI og MISO. Slave Select bruges bruges til at vælge hvilken slave der skal overføres data med. Den kan også sammenlignes med en chip select. SCLK bruges som clock frekvens, som synkronisere Salve select, MOSI og MISO. MOSI står for Master Output Slave Input, det er dataet master sender til slaven, ligeså er MISO Master Input Slave Output, som er dataet som, master modtager fra slaven.
Når der er flere end en slave, deler de SCLK, MOSI og MISO, men de har en Slave Select hver. Forløbet er sådan at master sender et SS signal til den ønskede slave. De slaver der ikke har modtaget et SS signal ingnorer det der bliver sendt på MOSI og MISO. I et system kun med en slave kan SS udelades, da der kun er en slave at vælge. 

\subsection{Overvejelser og valg}
Der er gået mange overvejelser i hvordan SPI'en mellem Microcontrollren og FPGA'en skal  opbygges.
Det blev vedtaget at Micocontrolleren var masteren og FPGA blev slaven, da FPGA'en helst skulle være så enkel som mulig.
Microcontrolleren har også indbygget tre SPI programmer, hvor den kan være masteren. De tre typer er TI Synchronous Serial Frame Format (SSFT), Freescale og Microwire. Til at overfører dataene mellem Microcontrolleren og FPGA’en skal der bruges en enkel protokol med full duplex. Hovedgrundene for at protokollen skal være enkel, er at der ikke sendes meget data, dataene er enkel og afstanden den sendes er kort. Derfor blev SSFT valgt som protokol. I SSFT liggre svaret på nogle af overvejelserne. Overvejelserne har været om MOSI skal trigger på rising eller falling edge, og hvordan Slave Select skulle opføre sig.   
Frekvensen af SCLK skal vælges så høj, at SPI ikke bliver en flaskehals i kommunikationen mellem Microcontroller og FPGA’en, og tilpas lav, så der ikke introduceres unødige fejlkilder, ved at FPGA’en ikke kan følge med. 
Størrelsen af framen, som sendes skulle være så stor at PWM (duty cycle?) kan sendes til FPGA’en og ticksne fra FPGA’en kan sendes i den rigtige opløsning. PWM og ticks har begge en størrekse af 11 bit, der skal også være et motor, retnings- og feedback bit.  Det giver at framen mindst skal være 14 bit lang. Microcontrolleren understøtter en størrelse mellem fire og 16 bit, 16 bit blev valgt da der så er plads til at udbygge.



\subsection{Brugen af SPI}


%%% FORMAT SPI

Fra Mikrokontroller til FPGA format:
\begin{figure}[th!]
\centering
\begin{tabular}{c|c|c|c|c}
1 motorbit &1 retningsbit & 1 set PWMbit & 2 ignorebits & 11 PWM dutycyclebits

\end{tabular}
\captionsetup{type=figure}
\caption[tekst i indholdsfortegnelsen]{tabeltekst}
\label{tb:}
\end{figure}




Fra FPGA til Mikrokontroller format:
\begin{figure}[th!]
\centering
\begin{tabular}{c|c|c|c}
1 motorbit & 4 ignorebits & 11 decoderbits

\end{tabular}
\captionsetup{type=figure}
\caption[tekst i indholdsfortegnelsen]{tabeltekst}
\label{tb:}
\end{figure}
