\section{Lerduens karakteristika}
\label{sec:udregning_af_parabel}

Jævnfør reglerne for ES, skal lerduen passere target crossing point (TCS) som er 
placeret i 4,57 [m] over origo med en fejlmargin på \(\pm\)0,45 [m] for passagen. 
Lerduen skal flyve 50 [m] - 52 [m]. "High House" er placeret 20,11 [m] fra TCS, i en 
højde af 3,05 [m]. 

\subsection{Parabel i 2 dimensioner}
For at simplificere udregningerne, bestemmes parablen først i 2D. Når denne er fundet kan den tredje dimension tilføjes.\\

Da luftmodstanden er negligerbar kan parablen (2. grads polynomium) findes ved at indsætte de kendte punkter. Kasteparablen er givet ved vektorfunktionen i ligning \ref{eq:pf:vektorparabel}.

\begin{equation}
	Pos(t) = \left( \begin{array}{c}
	x_{2D}(t) \\
	y_{2D}(t)
	\end{array}
	\right)
	= \left( \begin{array}{c}
	\cos \theta v_0 t + x_0 \\
	\sin \theta v_0 t - \frac{g}{2} t^2 + y_0
	\end{array}
	\right)
\label{eq:pf:vektorparabel}
\end{equation}

Hvor \(\theta\) er afskydningsvinklen, \(v_0\) er afskydningshastigheden, \(g=9,82 \left[ \frac { m }{ { s }^{ 2 } } \right] \) er tyngdeacceleration og \(x_0\), \(y_0\) er begyndelsespunktet. 

For at få en parabel på formen y(x), isoleres t i x(t) med henblik på at substituere t i y(x): (\(x_0\) sættes til 0.)

\begin{equation}
t = \frac{x}{\cos \left( \theta \right) v_0}
\label{eq:pf:x(t)}
\end{equation}

Den fundne værdi for t indsættes i \(y(t)\) og udtrykket reduceres, \citep[Side. 67]{fund_of_physics}.
\begin{align}
\begin{split}
y(t(x)) &= \sin \left( \theta \right) \frac{x}{\cos \left( \theta \right) v_0} v_0 - \frac{g}{2} \left(\frac{x}{\cos \left( \theta \right) v_0}\right)^2 + y_0 \\
y(x) &= \tan \left( \theta \right) x - \frac{gx^2}{2(\cos \left( \theta \right) v_0)^2} + y_0
\label{eq:pf:y(x(t))}
\end{split}
\end{align}

Det ses at der er 3 ubekendte i ligningen. I reglerne for ES affyres lerduerne fra en højde af \(y_0\) = 3,05 [m].
\begin{equation}
y(x) = \tan \theta x - \frac{gx^2}{2(\cos \left( \theta \right)  v_0)^2} + 3,05
\label{eq:pf:y(x)2}
\end{equation}
I reglerne fremgår det at lerduens flugt passere TCS, som er placeret i en højde på 4,57 [m], 20,11 [m] fra HH. Det sidste krav er at lerduen først skal ramme jorden efter 50 [m] - 52 [m], (52 [m] i projektet). \\
Det giver koordinatsættene (20,11 ; 4,57) og (52 ; 0). Vha. disse bestemmelser er udtrykkes \(\theta\) og \(v_0\).

\begin{eqnarray}
\theta &=& 9,103 \degree \\
v_0 &=& 34,589 \left[ \frac { m }{ s }  \right] 
\end{eqnarray}

 \(\theta\) og \(v_0\) indsættes i stedvektoren fra ligning \ref{eq:pf:vektorparabel} samt udtrykket fra ligning \ref{eq:pf:y(x)2}, hvorefter det reduceres.

\begin{align}
\begin{split}
	Pos(t) = \left( \begin{array}{c}
	x_{2D}(t) \\
	y_{2D}(t)
	\end{array}
	\right)
	&= \left( \begin{array}{c}
	\cos \left(9,103 \degree \right) 34,589 t \\
	\sin \left(9,103 \degree \right) 34,589 t - \frac{9,82}{2} t^2 + 3,05
	\end{array}
	\right) \\
% ------------------------------
	&= \left( \begin{array}{c}
	34,153 t \\
	- 4,91 t^2 + 5,473 t + 3,05
	\end{array}
	\right)
\label{eq:pf:vektorparabel2}
\end{split}
\end{align}
\todo[inline,author=Anders]{JEG MANGLER AT SKRIVE OMKRING SB..............}
Nedenstående ligning bruges til tidbestemmelsen a


Ligning bruges til bestemmelse af hvordn
\begin{align}
\begin{split}
y(x) &= \tan \left(9,103 \degree \right) x - \frac{9,82x^2}{2(\cos \left(9,103 \degree \right) 34,589)^2} + 3,05 \\
% -------------------------------
&= - 0,00421 x^2 + 0,1602 x  + 3,05
\label{eq:pf:y(x)3}
\end{split}
\end{align}


\subsection{Parabel i 3 dimensioner}
\label{subsubsec:para}
Nu omskrives parablen fundet i ligning \ref{eq:pf:vektorparabel2} til x,y,z koordinater. 
Grundplanet er xy og højden er givet ved z. Dvs. at \(z(t) = y_{2D}(t)\) og at \(x(t)\) samt \(y(t)\) afhænger af \(x_{2D}(t)\) samt vinklen \(\alpha\). 
\(\alpha\) er givet ved vinklen mellem parablen projekteret ned på xy-planet (som vist på figur \ref{fig:para_in_xy_plane}) og y-planet.
%\todo[inline, author=Michael, color=blue! 50]{Kan også skrives som vinklen mellem parablen og x,z planet?}
Stedvektoren for lerduens position i 3D er givet i nedenstående ligning, hvor \(\alpha = 15,872 \degree\).
\begin{align}
\begin{split}
Pos\left( t \right) = 
\left( \begin{matrix} x\left( t \right)  \\
 y\left( t \right)  \\ 
 z\left( t \right)  \end{matrix} \right) &=
 \left( \begin{matrix} - sin\left( \alpha  \right) \cdot { x }_{ 2D }\left( t \right) + 5.5 \\
 cos\left( \alpha  \right) \cdot { x }_{ 2D }\left( t \right) - 19.3  \\
  { y }_{ 2D }\left( t \right)  \end{matrix} \right)
\\
%----------------------------------------
&= \left( \begin{matrix} - 9,34\cdot t+5,5 \\
  32,851\cdot t-19,3 \\ 
 -{ 4,91\cdot t }^{ 2 }+5,473\cdot t+3,05\end{matrix} \right) 
\label{eq:pf:vektorparabel3d}
\end{split}
\end{align}




Bemærk udgangspunktet for kastet er flyttet fra origo, til HHs position. (Punkt D på figur \ref{fig:para_in_xy_plane}).
