\section{Matematisk model af Pan \& Tilt-systemet}
\label{sec:matPTS}
Under designet af en regulator skal systemet identificeres. Der er forskellige metoder til systemidentifikation.
Én metode ser som udgangspunkt systemet som en "black box", og identificerer systemet
på baggrund af sammenhørende værdier for input og output.
En anden metode tager udgangspunkt i matematisk at beskrive de enkelte dele i systemet
for på den måde at sammenstykke en model af hele systemet.
Det vælges at benytte den sidstnævnte metode til bestemmelsen af en simplificeret
model af Pan \& Tilt-systemet af følgende årsager:
\begin{itemize}
\item Matematiske modeller for systemets enkelte elementer findes i litteraturen.
\item De matematiske modeller er lineære og simple.
\item Den første metode vurderes mere tidskrævende.
\end{itemize}
Ulempen ved at bruge teoretiske lineære matematiske modeller er,
at de altid vil være tilnærmelser, og ikke beskriver ulineariteter som dødbånd og tidsforsinkelser.

En skitse af Pan \& Tilt-systemet findes i figur \todo{figur}.
\missingfigure{Model med Open Loop Pan\&Tilt-system (koblet)}

\subsection{Afkobling af pan og tilt}
Når den ene af de to rammer roterer vil en kraftmoment-induceret præcession påvirke den anden ramme.
Samtidig vil tilt-rammens vinkel påvirke inertimomentet omkring pan-aksen, og dermed
pan-motorens overførselsfunktion.
Der er altså tale om et Multi-Input Multi-Output (MIMO) system med en kobling mellem pan og tilt.
Det vælges at simplificere systemet til to Single-Input Single-Output (SISO) systemer, som gruppen er bekendt med.
Retfærdiggørelsen heraf ligger i, at præcessionens størrelse afhænger af vinkelaccelerationen, som er stærkt begrænset
for dette fysiske system, samt at tilt-rammens inertimoment er tilnærmelsesvis konstant, som yderligere beskrevet i afsnit \ref{sec:inertimoment}.
Afkoblingen af de to systemer, og opdelingen i hhv. pan og tilt gør, at der kan udvikles en separat regulator
til hvert system, og at der skal findes en overførselsfunktion for hvert system.
De følgende beregninger tager altså udgangspunkt i det simplificerede afkoblede system.

\subsection{DC-Motor}
To DC-motorer af typen EMG30 er forbundet til systemet.
Det antages, at motorerne kan beskrives ved samme matematiske model.
Den matematiske model af DC-motoren er beskrevet i appendix \ref{sec:dcmotor},
der også beskæftiger sig med bestemmelsen af parametrene for en EMG30-motor (det antages, at systemets motorer har samme parametre).
Motoren kan beskrives ved ligning \ref{eq:matVm_transient3}, hvor konstanterne \(k_1\), \(k_2\) og \(k_3\)
er givet ved ligningerne \ref{eq:matkonstanter}. Der henvises til appendix \ref{sec:dcmotor}
for yderligere forklaring af modellen.
\begin{equation}
	V_m\left(t\right)=k_1\cdot{}\frac{\mathrm d^2}{\mathrm d t^2} \big(\omega\left(t\right) \big)
		+k_2\cdot{}\frac{\mathrm d}{\mathrm d t} \big(\omega\left(t\right) \big)
		+k_3\cdot{}\omega\left(t\right)
	\label{eq:matVm_transient3}
 \end{equation}
\begin{equation}
	k_1=\frac{L_m\cdot{}\left(J_L+J_m\right)}{K_t},
	k_2=\frac{R_m\cdot{}\left(J_L+J_m\right)+L_m\cdot{}B}{K_t},
	k_3=K_b+\frac{R_m\cdot{}B}{K_t},
	\label{eq:matkonstanter} 
 \end{equation}
\(L_m\) er motorens ækvivalente induktans, \(R_m\) dens resistans, \(J_m\) dens indre inertimoment,
\(K_t\) er kraftmomentproportionalitetskonstanten, \(K_b\) er proportionalitetskonstanten for den modelektromotoriske kraft,
mens \(B\) er den viskøse friktionskoefficient.
\begin{figure}[th!]
	\centering
	\begin{tabular}{r|l|l}
Parameter&Værdi&Enhed\\\hline
\(R_m\)&\(5,215\)&\([\Omega]\)\\
\(L_m\)&\(2,2\cdot{}10^{-3}\)&\(\text{[H]}\)\\
\(K_b\)&\(0,517\)&\(\left[ \text{V}\cdot\frac{\text{s}}{\text{rad} }\right] \)\\
\(K_t\)&\(0,517\)&\(\left[ \frac{\text{N}\cdot \text{m}}{\text{A}} \right] \)\\
\(B\)&\(0,00319\)&\(\left[  \text{N} \cdot \text{m} \cdot \text{s}\right] \)\\
\(J_m\)&\(8,26\cdot10^{-4}\)&\(\left[ \text{kg}\cdot{\text{m}^2} \right]  \)\\
%\item \(\left| T _f \right|\)&\(0,0571\)&\(\left[ \text{N} \cdot \text{m} \right]  \)\\
\end{tabular}
	\captionsetup{type=table}
	\caption[Motorparametre]
			{Eksperimentelt bestemte motorparametre.}
	\label{tb:matmotorparametre}
\end{figure}
De eksperimentelt bestemte motorparametre står i tabel \ref{tb:matmotorparametre}.
Det er vigtigt at bemærke, at de to motorer adskiller sig fra hinanden ved parameteren \(J_L\), som er belastningens
inertimoment. Dette er ikke en egentlig motorparameter, men en parameter der udelukkende afhænger
af pan- og tilt-rammernes dimensioner og vinkel.
Vi er interesserede i en overførselsfunktion for DC-motoren. Med antagelse om at \(\omega\) (vinkelhastigheden) og dens tidsafledte er 0
Laplace-transformeres ligning \ref{eq:matVm_transient3}, og den resulterende overførselsfunktion findes
i ligning \ref{eq:transOmega}.
\begin{equation}
	\frac{\Omega\left(s\right)}{V_m\left(s\right)}=\frac{1}{k_1\cdot{}s^2+k_2\cdot{}s+k_3}
	\label{eq:transOmega}
 \end{equation}
Ligning \ref{eq:transOmega} beskriver motorens rotors vinkelhastighed \(\Omega\left(s\right)\) som funktion af spændingsfaldet
\(V_m\left(s\right)\) over motoren. Bemærk at reduktionsgearingen i motoren gør at vinkelhastigheden set fra Pan \& Tilt rammerne
er lavere.
Vinkelhastigheden \(\omega\) er den tidsafledte af vinklen \(\theta\),
og hvis det antages at motorens startvinkel er 0, så kan motorens overførselsfunktion
findes blot ved tilføjelse af en integrator \(\frac{1}{s}\) til ligning \ref{eq:transOmega}.
Motorens overførselsfunktion \(G_m\left(s\right)\) fra input spændingsfald til output vinkel (inden reduktionsgearing) findes
i ligning \ref{eq:transTheta}.
\begin{equation}
	G_m\left(s\right)=\frac{\Omega\left(s\right)}{V_m\left(s\right)}\cdot{}\frac{1}{s}=\frac{1}{k_1\cdot{}s^3+k_2\cdot{}s^2+k_3\cdot{}s}
	\label{eq:transTheta}
\end{equation}
DC-motorens overførselsfunktion afhænger af dens belastning,
og belastningen (inertimomentet \(J_L\) der skal roteres) er indeholdt i konstanterne \(k_1\) og \(k_2\).
Der er altså en separat overførselsfunktion \(G_m\left(s\right)\) for Pan og for Tilt,
da rammerne har forskelligt inertimoment.
De to belastningsinertimomenter benævnes hhv. \(J_{pan}\) og \(J_{tilt}\),
og de to forskellige overførselsfunktioner benævnes hhv. \(G_{m,p}\) og \(G_{m,t}\).
Bestemmelsen af inertimomenterne findes i afsnit \ref{sec:inertimoment}.

\subsection{FPGA-modulerne}
FPGA'ens positionsencoders læser motorens vinkel inden reduktionsgearingen,
altså outputtet af \(G_m\left(s\right)\).
Der er altså tale om en kvantisering af output-vinklen.
Gearingen til rammerne er i forholdet 1:3, og da der er 360 encoder-ticks pr. rotation [\cite{emgmotor}],
er kvantiseringsintervallet \(\frac{1}{1080}\).

På FPGA'en genereres et PWM-signal til hver motors H-bro ud fra en
duty cycle som mikrokontrolleren leverer.
PWM-signalet forstærkes af H-broen til et firkantsignal med en amplitude på 12 [V],
som leveres til motoren.
En overførselsfunktion for denne duty cycle - motorspændingsfald konvertering ønskes
til reguleringen.
Det vælges at modellere konverteringen som en simpel lineær forstærkning af en duty cycle mellem -100 \% og +100 \%
til en DC-spænding mellem -12 [V] og +12 [V]. Med denne simple model ændres systemets orden ikke,
og det vurderes, at tidsforsinkelsen fra input duty cycle til motorbevægelse er negligerbar.
Overførselsfunktionen fra duty cycle til DC-spænding står i ligning \ref{eq:transPWM}.
\begin{equation}
	PWM\left(s\right)=PWM=12
	\label{eq:transPWM}
\end{equation}
\todo[inline,author=Mikael,color=red]{Er det bedre at modellere konverteringen som et analogt lavpasfilter?????}
\todo[inline,author=Mikael]{Dedikér et underafsnit til diskussion af tidsforsinkelser, digitale såvel som analoge.}

\subsection{Pan \& Tilt-rammernes inertimoment}
\label{sec:inertimoment}
For at opnå den optimale respons, skal reguleringen tage højde for, at Pan \& Tilt-motorerne møder modstand i form af
Pan \& Tilt-rammernes inertimoment. Inertimomentet kan bestemmes ved kendskab til rammernes
masse og dimensioner. Beregningen tager udgangspunkt i den simplificerede model skitseret i figur \ref{fig:inerti_PTS}.
På figuren er rammernes enkelte længder markeret, og de er målt til \({L_{1}} =0,292\) [m],
\({L_{2}} =0,280\) [m], \({L_{3}}= 0,42\) [m], \({L_{4}} =0,246\) [m], \({L_{pro}}=0,04\) [m].
\begin{figure}[!th]
\centering
\begin{tikzpicture}[scale=0.8]
\include*{./graphics/inerti_PTS}
\end{tikzpicture}
\caption[Skitse af Pan \& Tilt-rammerne]{Skitse af Pan \& Tilt-rammerne.}
\label{fig:inerti_PTS}
\end{figure}

I appendix \ref{sec:inertimomentberegning} beregnes Tilt-rammens inertimoment omkring tilt-aksen,
Pan-rammens inertimoment omkring pan-aksen,
og Tilt-rammens påvirkning på inertimomentet omkring pan-aksen diskuteres.

Inertimomenterne er givet ved ligningerne \ref{eq:matinerti_tilt_pan_fak},
som beregnet i appendix \ref{sec:inertimomentberegning}.
\begin{align}
\label{eq:matinerti_tilt_pan_fak}
\begin{split}
{J_{tilt}}&=1,57499\cdot{10}^{-4} \text{ [kg m$^2$]}
\\
{J_{pan}}&=4,62172\cdot{10}^{-4} \text{ [kg m$^2$]}
\end{split}
\end{align}
\(J_{pan}\) er beregnet ud fra et konstant bidrag fra tilt-rammen under antagelsen om,
at denne står tilnærmelsesvis lodret under hele bevægelsen.

\subsection{Pan- og tilt-systemernes åbensløjfeoverførselsfunktioner}
Hvert SISO-undersystem består af én PWM-blok på FPGA'en og en DC-motor med belastning.
To overførselsfunktioner \(G_{pan}\) og \(G_{tilt}\) for pan- og tilt-systemet kan opskrives
som produktet af ligning \ref{eq:transPWM} og \ref{eq:transTheta}. De to overførselsfunktioner
står i ligningerne \ref{eq:transpantilt0}, og de er illustreret i figur \ref{fig:openloop1}.
\begin{align}
\label{eq:transpantilt0}
\begin{split}
	G_{pan}\left(s\right)&=PWM\left(s\right)\cdot{}G_{m,p}\left(s\right)\\
	&=12\cdot{}\frac{1}
			{\left(\frac{L_m\cdot{}\left(J_{pan}+J_m\right)}{K_t}\right)\cdot{}s^3
			+\left(\frac{R_m\cdot{}\left(J_{pan}+J_m\right)+L_m\cdot{}B}{K_t}\right)\cdot{}s^2
			+\left(K_b+\frac{R_m\cdot{}B}{K_t}\right)\cdot{}s}
	\\
	\\
	G_{tilt}\left(s\right)&=PWM\left(s\right)\cdot{}G_{m,t}\left(s\right)\\
	&=12\cdot{}\frac{1}
			{\left(\frac{L_m\cdot{}\left(J_{tilt}+J_m\right)}{K_t}\right)\cdot{}s^3
			+\left(\frac{R_m\cdot{}\left(J_{tilt}+J_m\right)+L_m\cdot{}B}{K_t}\right)\cdot{}s^2
			+\left(K_b+\frac{R_m\cdot{}B}{K_t}\right)\cdot{}s}
\end{split}
\end{align}
\begin{figure}[!th]
\centering
\begin{tikzpicture}[auto, node distance=2.6cm,>=latex']
\include*{./graphics/openloop1}
\end{tikzpicture}
\caption[Åbensløjfeoverførselsfunktioner]{Åbensløjfeoverførselsfunktionerne \(G_{pan}\left(s\right)\) og \(G_{tilt}\left(s\right)\).
	Signalet \(d.c.\left(t\right)\) er en duty cycle mellem -100 \% og +100 \%.}
\label{fig:openloop1}
\end{figure}
\todo[inline,author=Mikael]{Jeg vil gerne bede TikZ-sagkyndige om at godkende figur \ref{fig:openloop1}.\\Det er mit første værk.}

Åbensløjfeoverførselsfunktionerne \(G_{pan}\) og \(G_{tilt}\) danner udgangspunktet
for analysen der hører til designet af reguleringssløjferne.
