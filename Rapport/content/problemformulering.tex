\section{Problemformulering}
Tracking af en lerdue i English Skeet \todo{eventuelt starte med en liste over forkortelser}(ES) vha. PTS. 
Lerduen er i vores opstilling udskiftet med en bold, men ellers følges reglerne for ES
\todo[color=yellow! 100]{Henvisning til litteraturliste eller fodnote}. 
Bolden bevæger sig i et 3D rum med negligerbar luftmodstand. \\

\begin{figure}[th!]
\centering
\includegraphics[width=0.4\textwidth]{./graphics/skeet-diagram_med_akser}
\caption[tekst i indholdsfortegnelsen]{figurtekst}
\label{fig:}
\end{figure}	
Systemets input er 60 kartesisiske koordinatsæt (x,y,z) per sekund. Hvor disse 
koordinater stammer fra, er uden for projektafgrænsningen. \\

Systemet dimensioneres til brug ifb\todo{Den forkortelse kender jeg ikke} en konkurrence i ES. En skitse over 
banen ses på figur 1\todo{reference}. Det kartesiske koordinatsystem har origo i halvcirklens centrum, 
som indtegnet\todo{De er så indtegnet forkert.. Flot arbejde, klaphat..} 
nederst på figuren. I ES skal rammes serier af duer/mål der afskydes fra 
enten ”High House” eller ”Low House”, og med skud fra hver af de otte stationer langs 
cirkelperiferien. I dette projekt kigges kun på tilfældet hvor der afskydes mål fra ”High-
House” med pan-and-tilt systemet placeret på station 4.\\

Jævnfør reglerne for ES, skal duer/mål passere target crossing point (TCS) som er 
placeret i $4,57 [m]$ over origo. (0; 0; 4.57). Fejlmargin for passagen er $\pm0,45 [m]$. 
Målene skal desuden flyve $50 – 52 [m]$.  High house er placeret $20,11 [m]$ fra TCS, i en 
højde af $3,05 [m]$. Da luftmodstanden er negligerbar kan parablen (2. grads 
polynomium) findes ved at indsætte de kendte punkter. Herunder er parablen vist i et 
2D plan, figur\todo{reference}:
\begin{figure}[th!]
\centering
\includegraphics[width=0.4\textwidth]{./graphics/high_house_2D_parabola}
\caption[tekst i indholdsfortegnelsen]{figurtekst}
\label{fig:}
\end{figure}

Målet bliver affyret med en hastighed på $34 [m/s]$, i en vinkel på $9^{\circ}$ ift. xy-
planet. Set ovenfra bevæger målet sig som set på figur 3\todo{reference}. Figur 3\todo{reference} kan laves med 
GeoGebra…  Målet bevæger sig fra High-House til nedslagspunktet I, i alt $52 [m]$. 
PTS er placeret i punktet B. 


\begin{figure}[th!]
\centering
\includegraphics[width=0.4\textwidth]{./graphics/parabola_in_xy_plane}
\caption[tekst i indholdsfortegnelsen]{figurtekst}
\label{fig:}
\end{figure}